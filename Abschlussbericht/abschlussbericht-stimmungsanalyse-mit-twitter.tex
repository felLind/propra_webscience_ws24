\documentclass[researchlab,group,]{AIGpaper}

%%%% Package Imports %%%%%%%%%%%%%%%%%%%%%%%%%%%%%%%%%%%%%%%%%%%%%%%%%%%%%%%%%%%%%%%%%
\usepackage{graphicx}					    % enhanced support for graphics
\usepackage{tabularx}				      	% more flexible tabular
\usepackage{amsfonts}					    % math fonts
\usepackage{amssymb}					    % math symbols
\usepackage{amsmath}					    % overall enhancements to math environment
\usepackage{multirow}					    % multirow for tables
\usepackage{booktabs}					    % improved table design

%%%% optional packages
\usepackage{tikz}                           % creating graphs and other structures
\usepackage{glossaries}                     % glossaries package for glossary entries
\usepackage{soul}                           % for highlighting text (temporary usage)
\usepackage{algorithm}                      % for creating pseudo-code
\usepackage{algpseudocode}                  % for creating pseudo-code
\usepackage{multicol}                       % for creating multi-column environments
\usepackage{enumitem}                       % for reducing space in lists
\usepackage{hyphenat}                       % for hyphenation
\usepackage[hidelinks]{hyperref}            % for hyperlinks


\setlist{noitemsep}

\newcommand{\redhl}{\sethlcolor{red}\hl}
\newcommand{\greyhl}{\sethlcolor{lightgray}\hl}


%%%% Author and Title Information %%%%%%%%%%%%%%%%%%%%%%%%%%%%%%%%%%%%%%%%%%%%%%%%%%%
\author{Anne Huber, Andreas Franke, Felix Lindner,\\ Burak Özkan, Milomir Soknic}

\title{Stimmungsanalyse mit Twitter}

%%%% Glossary %%%%%%%%%%%%%%%%%%%%%%%%%%%%%%%%%%%%%%%%%%%%%%%%%%%%%%%%%%%%%%%%%%%%%%

\newacronym{bert}{BERT}{\textit{Bidirectional Encoder Representations from Transformers}}
\newacronym{cnn}{CNN}{\textit{Convolutional Neural Network}}
\newacronym{dl}{DL}{\textit{Deep Learning}}
\newacronym{knn}{k-NN}{\textit{k-Nearest Neighbors}}
\newacronym{llm}{LLM}{\textit{Large Language Model}}
\newacronym{lr}{LR}{Logistische Regression}
\newacronym{lstm}{LSTM}{\textit{Long Short-Term Memory}}
\newacronym{ml}{ML}{Maschinellen Lernens}
\newacronym{nb}{NB}{Naive Bayes}
\newacronym{nltk}{NLTK}{\textit{Natural Language Toolkit}}
\newacronym{roberta}{RoBERTa}{\textit{Robustly optimized BERT approach}}
\newacronym{sgd}{SGD}{\textit{Stochastic Gradient Descent}}
\newacronym{svm}{SVM}{\textit{Support Vector Machine}}
\newacronym{tfidf}{TF-IDF}{\textit{Term Frequency-Inverse Document Frequency}}

%%%% Abstract %%%%%%%%%%%%%%%%%%%%%%%%%%%%%%%%%%%%%%%%%%%%%%%%%%%%%%%%%%%%%%%%%%%%%%

\germanabstract{
    Dieser Bericht beschreibt die Durchführung und Ergebnisse eines Projekts zur Stimmungsanalyse von Tweets.
    Ziel des Projekts war es, die Effektivität verschiedener maschineller Lernverfahren, einschließlich klassischer Ansätze und \textit{Deep-Learning}-Ansätze, bei der Klassifizierung der Stimmung von Tweets zu untersuchen.
    Dazu wurden verschiedene Vorverarbeitungsschritte und Vektorisierungsverfahren für klassische Ansätze angewandt und evaluiert.
    Für die \textit{Deep-Learning} Ansätze wurden \textit{Finetuning}-Ansätze für BERT-basierte Modelle angewandt.
    Des Weiteren wurden für \textit{DeepSeek}-R1-Modelle Anfrage-basierte Ansätze mit und ohne Berücksichtigung von Aspekten untersucht.
    Die Ergebnisse zeigen, dass \textit{Deep-Learning}-Modelle, insbesondere BERT-basierte Modelle mit \textit{Finetuning} sowie aktuelle LLMs, eine höhere Genauigkeit bei der Stimmungsanalyse erzielen als klassische Ansätze.
}


\begin{document}

\maketitle % prints title and author information, as well as the abstract


% ===================== Beginning of the actual text section =====================

\section{Einleitung}

Twitter zählt zu den bedeutendsten sozialen Netzwerken, auf dem täglich Millionen von Nutzenden ihre Meinungen zu unterschiedlichen Themen äußern.
Aufgrund der enormen Menge an Tweets bietet die Plattform wertvolle Einblicke in die öffentliche Meinung und gesellschaftliche Diskurse \cite{pak2010twitter}.
Die Stimmungsanalyse, ein Teilgebiet der \gls{cl}, ermöglicht es, Emotionen und Meinungen automatisiert zu erkennen und zu kategorisieren.
Insbesondere im Bereich der Web Science trägt diese Methode dazu bei, Dynamiken in sozio-technischen Systemen besser zu verstehen.

Im Rahmen dieses Projekts werden verschiedene Verfahren des maschinellen Lernens auf einen Twitter-Datensatz angewendet, um die mittels Tweets ausgedrückte Stimmung der Nutzer zu analysieren.
Als Datengrundlage dient der \textit{Sentiment140}-Datensatz der auf \textit{HuggingFace}\footnote{https://huggingface.co/datasets/stanfordnlp/sentiment140} verfügbar ist.
Der Datensatz entält eine große Menge annotierter Tweets und wurde von Go et al. erstellt \cite{go2009twitter}.\\
Ziel der Analysen ist es, unterschiedliche Ansätze, einschließlich Ansätze aus dem Bereich \gls{dl}, hinsichtlich ihrer Fähigkeit zur Stimmungsanalyse zu evaluieren.

Im Bericht wird zunächst die Aufgabenverteilung im Team sowie die genutzten Tools und Methoden beschrieben.
Anschließend werden die Schritte der Datenaufbereitung, die eingesetzten Modelle und die durchgeführten Experimente erläutert.
Abschließend erfolgt eine Diskussion der Ergebnisse, gefolgt von einem Ausblick auf mögliche Optimierungen und weiterführende Forschungsansätze.


\section{Aufgabenverteilung}\label{sec:aufgabenverteilung}
% Ein Abschnitt pro Teammitglied
% Kurze Übersicht, was die Person innerhalb des Praktikums beigetragen hat, insbesondere
% entwickelter Code, Beitrag zum Abschlussbericht, organisatorischer Beitrag, Beitrag zur
% Abschlusspräsentation, etc.

\begin{multicols}{2}
[
Die Themen Literaturrecherche und Ideenfindung, Überprüfung der Datenqualität des gegebenen Datensatzes, Suche nach einem neuen Datensatz und die explorative Datenanalyse des \textit{Sentiment140} Datensatzes wurden von allen Projektmitgliedern bearbeitet.
Darüber hinaus wurden von allen Projektmitgliedern ein klassischen Verfahren angewandt und unterschiedliche Kombinationen von Vorverarbeitungsschritten in Bezug auf die Genauigkeit getestet.
Außerdem wurde die Zwischen- und Abschlusspräsentation nach der initialen Erstellung gemeinsam überarbeitet.
Der Abschnitte des Abschlussberichts wurden initial von einzelnen Mitgliedern geschrieben und anschließend gemeinsam überarbeitet.\\
Im folgenden sind die Aufgaben aufgezählt, die nicht von allen Mitgliedern durchgeführt wurden.\\
\redhl{systematischere Aufzählung der Aufgaben}\\
\redhl{Verweis auf Abschnitte/Unterabschnitte mit \textit{ref} und \textit{label}?}
]

\subsection{Anne Huber}
\begin{itemize}
    \item Moderator der Gruppe
    \item Kommunikation nach Außen
    \item Schreiben der Protokolle
    \item Halten der Zwischenpräsentation
    \item Klassische Verfahren: \gls{knn}
    \item Abschlussbericht: 2. Aufgabenverteilung, 3. Teaminterne Organisation, 7. Ausblick
\end{itemize}


\subsection{Andreas Franke}
\begin{itemize}
    \item Klassische Verfahren: \gls{svm}
    \item Erstellung eines Frameworks zur Durchführung und zum Vergleich der klassischen Ansätze
    \item Durchführung des \textit{Zero-Shot}-Ansatzes (\textit{DeepSeek} R1 1.5B, 32B)
    \item Abschlussbericht: 4. Datensätze und Problemstellung, 5.1.2 SVM, 6. Experimente
\end{itemize}

\subsection{Felix Lindner}
\begin{itemize}
    \item Klassische Verfahren: Entscheidungsbäume/Entscheidungswälder
    \item Refactoring und Vereinheitlichung des Frameworks der klassischen Verfahren
    \item Anwendung des \textit{Deep Learning} Ansatzes (\textit{DistilBERT}, \textit{RoBERTa})
    \item Durchführung des \textit{Zero-Shot}-Ansatzes (\textit{DeepSeek} R1 70B)
    \item Halten der Abschlusspräsentation
    \item Abschlussbericht: 5.2 Deep Learning
\end{itemize}

\subsection{Burak Özkan}
\begin{itemize}
    \item Klassische Verfahren: Naiver Bayes Klassifikator
    \item Finden des Datensatzes \textit{Sentiment140}
    \item Erstellen der finalen Abschlusspräsentation
    \item Abschlussbericht: 5.1.3 Naiver Bayes Klassifikator, 8. Zusammenfassung und Fazit
\end{itemize}

\subsection{Milomir Soknic}
\begin{itemize}
    \item Klassische Verfahren: Logistische Regression
    \item Erstellen der Zwischenpräsentation
    \item Erstellen der initialen Abschlusspräsentation
    \item Abschlussbericht: 5.1.1 Logistische Regression, 1. Einleitung
\end{itemize}

\end{multicols}


\section{Teaminterne Organisation}\label{sec:teaminterneorganisation}
% Wie wurde innerhalb des Teams kommuniziert?
% Welche Programmiersprache? Warum?
% Welche Tools/Techniken wurden verwendet?
% etc.

Zur asynchronen Kommunikation der Projektmitglieder wurde eine private Lerngruppe in einer Discord-Instanz der Studenten der FernUniversität Hagen eingerichtet.
Dort wurden auch die regelmäßigen Treffen über die vorhandenen Sprachkanäle durchgeführt.
Die Intervalle der Treffen wurden in Absprache mit allen Mitgliedern und in Abhängigkeit von den durchgeführten Schritten festgelegt.
In den Treffen wurden die erreichten Ergebnisse besprochen sowie mögliche nächste Schritte diskutiert und geplant.

Als Programmiersprache wurde \textit{Python} gewählt, da bei allen Projektmitgliedern unter anderem durch die vorherige Teilnahme am Modul \textit{Einführung in Maschinelles Lernen} Vorkenntnisse vorhanden waren und viele Bibliotheken für maschinelles Lernen und \textit{Deep Learning} in \textit{Python} verfügbar sind.
Zur Verwaltung der verwendeten Python-Pakete und zur Vereinheitlichung und Prüfung der erstellten Dateien wurde \textit{Poetry} \cite{poetry2025} sowie \textit{pre-commit}\cite{precommit2025} verwendet.

Für die klassischen \gls{ml}-Verfahren wurde insbesondere die Bibliothek \textit{scikit-learn} eingesetzt.
Darüber hinaus wurde für die Anwendung der \gls{dl} Ansätze und die eigenen Ansätze die \textit{transformers} Bibliothek von \textit{Hugging Face} genutzt.

Zur Verwaltung der Projektdateien und der gemeinsamen Arbeit wurde ein \textit{GitHub Repository} \cite{githubrepo2025} eingerichtet.
Für neue Ansätze und Ergebnisse, sowie Änderungen an Code und Dokumenten, wurden in \textit{GitHub} \textit{Branches} angelegt und über \textit{Pull-Requests} überprüft und diskutiert.

Die Abfolge der einzelnen Projektschritte orientierte sich an dem \textit{Data Science Life Cycle} (vgl. \cite[Abb. 2]{Stodden2020}).
Dieser wurde während der Anwendung der klassischen Methoden einmal durchlaufen.
Dies beinhaltet die wiederholte Durchführung von Schritten, wie \textit{Obtain Data} und \textit{Data Exploration} oder die parallele Ausführung von \textit{Data Preparation} und \textit{Model Estimation} bei der Durchführung der unterschiedlichden klassischen Verfahren.
Im Anschluss, bei der Anwendung der \textit{Deep Learning} Modelle, wurde gleichermaßen mehrfach über gewisse Schritte des \textit{Data Science Life Cycle} iteriert.


\section{Datensätze und Problemstellung}\label{sec:datensätzeundproblemstellung}

\subsection{Datensätze}

Es wurden die separaten Trainings- und Testdatensätze des \textit{Sentiment140}-Datensatzes verwendet.

\subsubsection{Trainingsdatensatz}\label{subsec:trainingsdatensatz}
Der verwendete Datensatz \textit{Sentiment140} enthält 1,6 Mio. modifizierte Tweets.
Die Tweets wurden im Zeitraum von April 2009 bis Juni 2009 erstellt.
Jedem Tweet ist eine binäre Stimmungsklasse (positiv/negativ) zugeordnet.

Die Tweets wurden mithilfe der Twitter-API gesammelt, indem nach Tweets gesucht wurde, die bestimmte Emoticons mit positiver oder negativer Bedeutung enthalten (siehe Tabelle 3 im Artikel zum Datensatz \cite[S. 4]{go2009twitter}).
Anhand der verwendeten Emoticons wurden die Tweets in Klassen mit positiver und negativer Stimmung eingeteilt.
Tweets, die sowohl positive als auch negative Emoticons enthalten, sind nicht im Datensatz enthalten.
Go et al. \cite{go2009twitter} weisen darauf hin, dass die Stimmungsklassen nicht fehlerfrei sind (\cite[Abschnitt 2.2]{go2009twitter}) und es sich um \textit{Noisy Labels} handelt.
Als \textit{Noisy Labels} werden Klassenzuordnungen bezeichnet, die nicht fehlerfrei sind und die Qualität eines Datensatzes beeinträchtigen können \cite{song2022learning}.
Für beide Stimmungsklassen enthält der Datensatz jeweils 800.000 Einträge.

Die Tweet-Texte im Datensatz wurden so angepasst, dass die Emoticons, die zur Einteilung verwendet wurden, entfernt wurden.
Weiterhin enthält der Datensatz keine Retweets und keine Duplikate.

\subsubsection{Testdatensatz}\label{subsec:testdata}

Von Go et al. \cite{go2009twitter} wurde ein Testdatensatz mit 498 Tweets erstellt, der 177 Tweets mit positiver Stimmung und 182 Tweets mit negativer Stimmung enthält.
Die restlichen Tweets sind als neutral klassifiziert und wurden bei der Evaluierung der Modelle nicht berücksichtigt.
Die Tweets wurden über die Twitter-API durch Anfragen mit spezifischen \textit{Query}-Ausdrücken ausgewählt und manuell mit Stimmungsklassen versehen.
Die \textit{Query}-Ausdrücke sind in Tabelle 4 des Artikels von Go et al. \cite[S. 5]{go2009twitter} aufgeführt und sind im Testdatensatz enthalten.
Die Tweets enthielten nicht in jedem Fall Emoticons.

\subsection{Problemstellung}

Im Rahmen der Stimmungsanalyse für Twitter wird versucht, die Tweets in zwei oder mehr Klassen einzuteilen \cite{zimbra2018state}.

Die Analyse der Stimmung von Tweets wird allerdings als besonders herausfordernd angesehen \cite{agarwal2011sentiment, giachanou2016like, zimbra2018state}.
Giachanou und Crestani \cite{giachanou2016like} nennen neben der Längenbeschränkung von Tweets auf 140 Zeichen (bzw. 280 seit 2017) insbesondere die informelle Art von Tweets als Herausforderung in Bezug auf Stimmungsanalysen.
Agarwal et al. \cite{agarwal2011sentiment} und Zimbra et al. \cite{zimbra2018state} weisen darauf hin, dass aufgrund der Längenbeschränkung besonders häufig Abkürzungen, Emoticons und andere Zeichen mit spezieller Bedeutung oder Umgangssprache in Tweets verwendet werden.

Das Ziel dieser Arbeit ist es, bestehende Methoden zur automatischen Stimmungsanalyse von Tweets anzuwenden und zu evaluieren.
Dabei sollen sowohl klassische \gls{ml}-Ansätze als auch moderne \gls{dl}-Ansätze untersucht werden.

Die zentrale Problemstellung lautet: Wie effektiv sind verschiedene \gls{ml}-Ansätze bei der Stimmungsanalyse von Tweets?

Insbesondere soll untersucht werden, welche Ansätze die besten Ergebnisse in Bezug auf die Genauigkeit der Klassifikation liefern.
Darüber hinaus sollen die Herausforderungen und Limitationen der Stimmungsanalyse von Tweets identifiziert und diskutiert werden.


\section{Ansätze}
% Welche klassischen, Deep Learning -, und eigenen Ansätze wurde verwendet?
% Kurze Beschreibung neuer Techniken und Ideen

\subsection{Klassische Ansätze}

Als klassische Ansätze zur Stimmungsanalyse von Tweets wurden folgende überwachten Lernverfahren ausgewählt, weil diese besonders häufig in der Literatur zu finden sind \cite{wankhade2022survey, medhat2014sentiment, zimbra2018state}:

\begin{itemize}
    \item Logistische Regression
    \item \gls{svm}
    \item Naiver Bayes Klassifikator
\end{itemize}

\redhl{TODO: Ggf. Anmerkung ergänzen, dass zwei weitere klassische Ansätze evaluiert wurden.}

\subsubsection{Logistische Regression}

Die logistische Regression ist ein Verfahren zur Klassifikation, das auf der Sigmoid-Funktion basiert. Diese Funktion ist definiert als:

\begin{equation*}
    \sigma(z) = \frac{1}{1 + e^{-z}}
\end{equation*}

wodurch Werte aus dem gesamten Zahlenraum auf das Intervall $(0,1)$ abgebildet werden. Die Entscheidungsregel für einen Datenpunkt $x$ ergibt sich durch:

\begin{equation*}
    h_{\theta}(x) = \sigma(\theta \cdot x + b), \text{ wobei } \theta \in \mathbb{R}^{n}, b \in \mathbb{R}
\end{equation*}

Ein Datenpunkt wird dabei der Klasse 1 zugeordnet, wenn $h_{\theta}(x) \geq 0.5$, ansonsten der Klasse 0:

\begin{equation*}
    clf_{\theta, b}(x) =
    \begin{cases}
        1, & \text{wenn } h_{\theta}(x) \geq 0.5 \\
        0, & \text{sonst}
    \end{cases}
\end{equation*}

Zur Bestimmung der Parameter $\theta$ und $b$ wird die logistische Verlustfunktion minimiert:

\begin{equation*}
    \min_{\theta, b} \frac{1}{m} \sum_{i=1}^{m} \left[-y_i \log h_{\theta}(x_i) - (1 - y_i) \log(1 - h_{\theta}(x_i))\right]
\end{equation*}

Dieses Optimierungsproblem ist konvex und wird mithilfe des \textit{Gradient Descent} gelöst. Dabei werden die Parameter iterativ durch:

\begin{equation*}
    \theta \leftarrow \theta - \alpha \nabla_{\theta} L
\end{equation*}

aktualisiert, wobei $\alpha > 0$ die Lernrate ist. Eine Regularisierung kann durch einen zusätzlichen Term $\lambda \lVert\theta\rVert^2$ eingeführt werden, um Überanpassung zu vermeiden.

\subsubsection{\textit{Support Vector Machine}}

Die \gls{svm} wird oftmals zur Klassifikation verwendet.
Die Klassifikation erfolgt dabei durch die Bestimmung einer Hyperebene, die die Daten in zwei Klassen trennt.
Eine Hyperebene in $\mathbb{R}^{n}$ ist definiert als die Menge aller Punkte $x\in\mathbb{R}^n$ für die gilt:
\begin{equation*}
    \theta \cdot x - b = 0, \text{ wobei } \theta \in \mathbb{R}^{n}, b\in\mathbb{R}
\end{equation*}

Für linear nicht separierbare Datensätze, lässt sich keine Hyperebene finden, die die Daten perfekt trennt.
Stattdessen wird versucht eine Hyperebene zu bestimmen, die die Daten unter Berücksichtigung der \textit{Hinge}-Fehlerfunktion\footnote{
    Die \textit{Hinge}-Fehlerfunktion ist definiert als:
    \begin{equation*}
        L^{hinge}(D, \theta, b) = \sum_{1}^{m}\max\lbrace0, 1 - y_i(\theta \cdot x_i - b)\rbrace
    \end{equation*}
} möglichst gut trennt.
Die optimalen Parameter zur Bestimmung der Hyperebene können durch folgendes Minimierungsproblem, für einen Regularisierungsparameter $C\geq0$, bestimmt werden:
\begin{equation*}
    \min_{w, b} \lvert\lvert \theta \rvert\rvert + C \frac{1}{m}\sum_{i=1}^{m} \max\lbrace0, 1 - y_i(\theta \cdot x_i - b)\rbrace
\end{equation*}

Die binäre Klassifikation von Datenpunkten erfolgt dann durch die Bestimmung der Klasse des Datenpunktes $x$ durch:
\begin{equation*}
    clf_{\theta, b}(x) =
    \begin{cases}
        1, & \text{wenn } \theta \cdot x - b \geq 0 \\
        -1, & \text{sonst}
    \end{cases}
\end{equation*}

\subsubsection{Naiver Bayes Klassifikator}
Der \textit{naive Bayes-Klassifikator}, benannt nach dem englischen Mathematiker Thomas Bayes, ist ein maschinelles Lernverfahren, das aufgrund seiner Einfachheit und Effizienz häufig für Klassifikationsprobleme eingesetzt wird \cite{wankhade2022survey, medhat2014sentiment, zimbra2018state}.

Das Ziel ist es, für einen \textit{Trainingsdatensatz} $D$ eine optimale \textit{Hypothese} $h$ zu finden. Das Verfahren basiert auf dem \textit{Bayes-Theorem}, welches uns ermöglicht, uns der gesuchten optimalen Wahrscheinlichkeit $P(h|D)$ anzunähern. Dabei ist $P(h|D)$ die Wahrscheinlichkeit von $h$, gegeben der Beobachtung $D$:

\begin{equation*}
    P(h|D) = \frac{P(D|h)P(h)}{P(D)}
\end{equation*}

Wir suchen also eine Hypothese $h^*$ die den Wert $P(h|D)$ maximiert:
\begin{equation*}
    h^* = \arg\max_{h} P(h|D)
\end{equation*}

Die Grundidee des naiven Bayes-Klassifikators ist die Annahme, dass die einzelnen Merkmale unabhängig voneinander sind.

Sei $D = (x^{(1)}, y^{(1)}), \dots, (x^{(m)}, y^{(m)})$ ein Datensatz, $c$ eine Klasse im Merkmalsraum $Z$  und $x=(x_1, \dots,x_n)\in Z_1 \times \dots \times Z_n$ ein neuer Datenpunkt. Dann ist der Naive-Bayes-Klassifikator
\begin{equation*}
    clf_D^{NaiveBayes}(x) = \arg \max_{c\in Z} P(c|D)P(x_1|c,D)P(x_2|c,D) \dots P(x_n|c,D)
\end{equation*}
mit
\begin{equation*}
    P(c|D) = \frac{|\{(z,c)\in D\}|}{|D|}
\end{equation*}
\begin{equation*}
    P(x_i|c,D) = \frac{|\{(z^\prime,c)\in D| z^\prime = (z_1, \dots, z_n), z_i=x_i \}|}{|\{(z,c)\in D\}|} \quad \text{für} \: i = 1, \dots, n
\end{equation*}

Trotz der naiven Unabhängigkeitsannahme führt dieses Verfahren in der Praxis für eine Vielzahl von Anwendungsfällen zu guten Ergebnissen \cite{hand2001idiot}. Der naive Bayes-Klassifikator wird daher neben der Sentimentanalyse häufig auch für andere Textklassifikationsaufgaben eingesetzt. Eine typische Anwendung ist z.B. die Spam-Filterung \cite{sahami1998bayesian}.

\subsection{Deep Learning Ansätze}
Als Deep Learning Ansätze haben wir zuerst zwei auf \gls{bert} \cite{devlin2018bert} basierende Modelle ausgewählt, welche wir durch \textit{Finetuning} mit dem Datensatz sentiment140 trainiert haben. 
Zum einen das auf Twitter-Daten vortrainierte Modell RoBERTa \cite{liu2019roberta} und zum anderen das von \gls{bert} destillierte Modell DistillBert \cite{sanh2019distilbert}. 
BERT-Modelle basieren auf der von Vaswani et al. beschriebenen Architektur \textit{Multi-layer Bidirectional Transformer Encoder} \cite{vaswani2017attention}.


\section{Experimente}

\subsection{Experimentaufbau}

Für die Experimente wurden unterschiedliche Schritte in Abhängigkeit der Verfahren durchgeführt.

\subsubsection{Klassische Ansätze}\label{subsubsec:experimente-klassische-ansaetze}

Für die klassischen Verfahren wurde neben der Genauigkeit der Klassifikation für unterschiedliche Modelle auch analysiert, inwiefern Vorverarbeitungsschritte und Tokeni\-sie\-rungs\-verfahren die Genauigkeit der Klassifikation beeinflussen.

\paragraph{Datenvorverarbeitung}
Im Rahmen der Datenvorverarbeitung wurden die verwendeten Stoppwörter und Normalisierungsverfahren variiert.

Nach \cite[S.27]{manning2009introduction} werden unter dem Begriff Stoppwörter Wörter verstanden, die einen geringen Informationsgehalt haben und deshalb aus Texten entfernt werden, wie beispielsweise \textit{und} oder \textit{oder}.
Es wurden drei verschiedene Verfahren zur Behandlung von Stoppwörtern verwendet: Beibehaltung aller Stoppwörter, Verwendung der vordefinierten \gls{nltk}-Stoppwortliste und Verwendung einer eigenen Stoppwortliste zur Entfernung spezifischer Stoppwörter.

Normalisierungsverfahren dienen dazu, die Worte oder Token in Texten zu vereinheitlichen \cite[S.28]{manning2009introduction}.
Es wurden drei verschiedene Verfahren zur Normalisierung der Worte verwendet: keine Normalisierung, Lemmatisierung (mit \textit{WordNet Lemmatizer}) und Stemming (mit \textit{Porter Stemmer}).

Das Training der Modelle wurde auf Basis der vorverarbeiteten Daten durchgeführt.
Im Rahmen der Vorverarbeitung der Daten wurden die Tweets bereinigt, die einzelnen Token normalisiert und abschließend die Stoppwörter entfernt.
In Abschnitt \ref{subsec:appendix-data-preparation} im Anhang ist der Algorithmus detaillierter beschrieben (s. Algorithmus~\ref{alg:data-preparation}) und ein Beispiel gegeben.

\paragraph{Training und Evaluation}
Auf Basis der vorverarbeiteten Daten wurden die Modelle trainiert und evaluiert.

Die Texte der vorverarbeiteten Tweets werden mittels Vektorisierungsverfahren in numerische Repräsentationen bzw. Vektoren transformiert.
Es wurden zwei Vekto\-risierungs\-verfahren verwendet: \gls{tfidf}-Vek\-to\-ri\-sie\-rung und \textit{Hash}-basierte Vektorisierung\footnote{Für beide Verfahren wurden die Implementierungen der \textit{scikit-learn} Bibliothek verwendet.}.
\gls{tfidf} Vektorisierung ist ein Verfahren zur Gewichtung von Termen in Texten, das die Häufigkeit der Terme in einem Dokument und die inverse Häufigkeit von Dokumenten mit diesem Termen berücksichtigt \cite[S. 119]{manning2009introduction}.
Die \textit{Hash}-basierte Vektorisierung ist ein Verfahren, das die Wörter in einem Dokument mittels einer Hash-Funktion in numerische Werte umwandelt, um die Wörter in einen Vektor fester Länge zu kodieren \cite{sklearnextraction2025}.

Für die verwendeten Vektorisierungsverfahren wurden unterschiedliche Konfigurationen von n-Grammen verwendet.
n-Gramme bezeichnen nach \cite[S.33]{jm3} eine Sequenz von $n$ aufeinanderfolgenden Wörtern.
Für die Vektorisierungsverfahren wurden Instanzen mit Kombinationen von 1-Grammen, 2-Grammen und 3-Grammen verwendet.

Während des Trainings werden die Modelle auf den Trainingsdaten trainiert und auf den Validierungsdaten evaluiert.
Die Genauigkeit der Modelle wird abschließend auf den Testdaten evaluiert.
Algorithmus~\ref{alg:model-training} beschreibt die Schritte für das Training und die Evaluation der Modelle.

\subsubsection{Deep Learning Ansätze} \label{subsubsec:experimente-deep-learning-ansaetze}

Für die \gls{dl}-Ansätze, bei denen \gls{bert}-basierte Modelle verwendet wurden, wurden die Standard-Tokenizer der \textit{Hugging Face} Modelle verwendet, sodass keine weiteren Vorverarbeitungsschritte oder Vektorisierungsverfahren durchgeführt wurden.

\paragraph{\textit{Finetuning} der \gls{bert}-Modelle}

Die \gls{bert}-basierten Modelle \textit{twitter-roberta\hyp{}base\hyp{}sentiment} und \textit{distilbert-base-uncased} wurden über die \textit{Transfomers} Python-Bibliothek von \textit{Hugging Face} zunächst mit Hilfe des Trainingsdatensatzes trainiert und ausgeführt.

Für das \textit{Finetuning} wurde die Standardkonfiguration der Bibliothek verwendet.
Variiert wurden die Datensatzgröße, also die Anzahl der Tweets, die für das \textit{Finetuning} verwendet wurden, und die Lernrate.
Die Werte für die Lernrate lagen zwischen $10^{-4}$ und $10^{-6}$ und die Datensatzgröße zwischen 2500 und 20000. Die Werte sind in Tabelle \ref{tab:dl-params} aufgeführt und orientieren sich an dem Vorgehen von Barbieri et al. \cite{barbieri2020tweeteval}.

\paragraph{Verwendung der \textit{DeepSeek}-Modelle}

Für den eigenen Ansatz haben wir zuerst versucht, die von \textit{DeepSeek}-R1 destillierten Modelle durch \textit{Finetuning} auf den Datensatz zu trainieren.
Dafür wurden den Modellen jeweils zwei zusätzliche voll-vernetzte Schichten angefügt.
Diese Architektur entspricht dem Aufbau des \textit{twitter-roberta-base-sentiment} Modells zur Klassifikation.

Die erste voll-vernetzte Schicht hatte als Ein- und Ausgabe die Dimensionen der letzten Schicht der Modelle.
Die zweite voll-vernetzte Schicht hatte als Eingabe die Ausgabe der ersten voll-vernetzten Schicht und als Ausgabe die Anzahl der Klassen.

Die Durchführung des \textit{Finetunings} war für das kleinste destillierte \textit{DeepSeek}-R1-Modell mit 1,5 Milliarden Parametern noch möglich.
Ab dem nächstgrößeren Modell (\textit{DeepSeek-R1:7B}) wurden die Hardware-Anforderungen zu groß\footnote{Vergleiche mit \gls{sgd}-Optimierung: 7 Mrd. Parameter $\times$ 2 Byte je Gewicht $\times$ 2 Byte je Gradient = 26 GB}.

Als weiteren Ansatz verwendeten wir die DeepSeek-R1-Modelle nur in der Ausführung und ließen die Stimmung per Prompt klassifizieren.
Die Modelle wurden mit \textit{Ollama} ausgeführt und per Pythonskript angefragt.
Die Anfragen wurden mit und ohne \textit{Query}-Ausdruck (siehe Abschnitt \ref{subsec:testdata}) durchgeführt.

Die Anfragen mit Query-Ausdruck hatten folgende Struktur (wobei die Platzhalter $Query\-Term$ und $Tweet$ durch die entsprechenden Werte ersetzt wurden):
\begin{quote}
    Tweet sentiment? Sentiment Topic: $QueryTerm$\\
    Answer with positive or negative. Provide reasoning in JSON.\\
    Tweet: $Tweet$
\end{quote}

\begin{table}
    \center
    \begin{tabular}{lccccc}
        \toprule
        Modell & Normalisierung & Stoppwortliste   & Anz. Merkmale & n-Gramme & Genauigkeit \\
        \midrule
        LR  & Porter  & eig. Liste & 250.000 & (1,2) & 0.859 \\
        SVM & Porter  & -          & maximal & (1,3) & 0.858 \\
        SVM & Porter  & eig. Liste & 50.000  & (1,2) & 0.858 \\
        LR  & WordNet & -          & maximal & (1,3) & 0.858 \\
        SVM & WordNet & -          & maximal & (1,3) & 0.858 \\
        \bottomrule
    \end{tabular}
    \caption{
        Top 5 Modelle nach Testgenauigkeit angeordnet (Mittelwerte von drei Ausführungen).
        Die Bezeichner $(1, k)$ in der Spalte \textit{n-Gramme} geben an, dass k-Gramme mit $k\in\lbrace1,\cdots,3\rbrace$ verwendet wurden.
    }
    \label{tab:top-5-models}
\end{table}

\subsection{Modell-Parameter und Evaluationsmetriken}\label{subsec:modell-parameter-und-evaluationsmetriken}

Für die ausgewählten klassischen Modelle wurden die Standard-Parameterwerte von \textit{scikit-learn} verwendet.
Für die \gls{bert}-basierten Modelle und die \textit{DeepSeek}-Modelle wurden die veröffentlichten Modelle verwendet bzw. auf diesen im Rahmen des \textit{Finetunings} aufgesetzt.

Nach~\cite{wankhade2022survey} werden für die Evaluierung von Klassifikationsmodellen vor allem das \textit{Genauig\-keits\-maß}, die \textit{Präzision} oder das \textit{F1-Maß} verwendet.
Die Klassenverteilung für die positive und negative Klasse der Trainingsdaten ist ausgeglichen und die Verteilung der Testdaten ist ebenfalls relativ ausgeglichen.
Aufgrund der einfachen Interpretierbarkeit wurde deshalb das \textit{Genauigkeitsmaß} als Evaluationsmetrik verwendet.

\subsection{Ergebnisse}

Eine Übersicht über die maximalen Genauigkeiten je Modell und Ansatz ist im Anhang in Diagramm~\ref{fig:results} dargestellt.

\subsubsection{Klassische Ansätze}\label{subsubsec:ergebnisse-klassische-ansaetze}

In Tabelle~\ref{tab:top-5-models} sind die Top 5 Modelle nach Testgenauigkeit sortiert aufgeführt.
Die maximal erzielte Genauigkeit beträgt $0,859$.

\paragraph{Sensitivität Modell}
In Tabelle~\ref{tab:stats-per-model} sind die Statistiken der Testgenauigkeit für die Modelle \gls{svm}, \gls{lr} und \gls{nb} über alle Parameter-Kombinationen aufgeführt.

Die \gls{lr}-Modelle erzielen im Mittel die höchsten Genauigkeiten über alle Parameter-Kombinationen.
Weiterhin ist die Standardabweichung der Genauigkeiten für die \gls{lr} mit $0,022$ am niedrigsten.

Modelle auf Basis der \gls{svm} erzielen im Mittel und im Median 1\% niedrigere Genauigkeiten.
Mit \gls{nb}-Modellen wurden im Mittel die niedrigsten Genauigkeiten erreicht, wobei die Standardabweichung am höchsten war.

\begin{table}
    \center
    \begin{tabular}{lccccc}
        \toprule
        & \multicolumn{5}{c}{Testgenauigkeit über alle Parameter-Kombinationen} \\
        Modell             & Mittelwert & Median & Std.-Abweichung & Minimum & Maximum \\
        \midrule
        LR                 & 0,819      & 0,822  & 0,022           & 0,727   & 0,861 \\
        SVM                & 0,809      & 0,813  & 0,024           & 0,730   & 0,858 \\
        NB                 & 0,778      & 0,784  & 0,049           & 0,685   & 0,852 \\
        \bottomrule
    \end{tabular}
    \caption{Statistiken der Testgenauigkeit für die Modelle \gls{lr}, \gls{svm} und \gls{nb}.}
    \label{tab:stats-per-model}
\end{table}

\paragraph{Sensitivität Umgang mit Stoppwörtern}

Über alle Parameter-Kombinationen hinweg ist die Genauigkeit im Mittel für die Datensätze mit Entfernung der Stoppwörter höher als für die Datensätze ohne Entfernung.
Weiterhin ist die Genauigkeit mit der eigens definierten Stoppwortliste höher als mit der \gls{nltk}-Liste.

\paragraph{Sensitivität Normalisierungsverfahren}

Für die unterschiedlichen Normalisierungsverfahren ergeben sich keine signifikanten Unterschiede in der Genauigkeit der Modelle.

\paragraph{Sensitivität Vektorisierungsverfahren und n-Gramme}

Für alle Modelle ist die Ge\-nauig\-keit höher, wenn das \gls{tfidf}-Vektorisierungsverfahren verwendet wird.

Für alle Modelle steigt die Genauigkeit mit der Anzahl an berücksichtigten Merkmalen und, wenn mehr als nur Unigramme berücksichtigt werden.
Die Verwendung von Uni- und Bigrammen führt im Mittel zu den höchsten Genauigkeiten.

\subsubsection{\textit{Deep Learning} Ansätze}\label{subsubsec:ergebnisse-deep-learning-ansaetze}

\paragraph{\textit{Finetuning} der \gls{bert}-Modelle}
Mit dem auf aktuelleren Twitter-Daten trainierten Modell \textit{twitter-roberta-base-sentiment} wurden Genauigkeiten von $0,83$ auf dem Testdatensatz erzielt.
Durch \textit{Finetuning} wurden Genauigkeiten von $0,922$ für das Modell \textit{twitter-roberta-base-sentiment} und $0,849$ für das Modell \textit{distilbert-base-uncased} erreicht.

Kleinere Lernraten führen zu höheren Genauigkeiten für das Modell \textit{twitter-roberta-base-sentiment}, während für das Modell \textit{distilbert-base-uncased} höhere Lernraten zu bes\-se\-ren Ergebnissen führen.
Dies ist vermutlich darauf zurückzuführen, dass das Modell \textit{twitter-roberta-base-sentiment} bereits auf Twitter-Daten trainiert wurde und das \textit{distilbert-base-uncased} Modell lediglich auf einem allgemeinen Korpus.

\paragraph{Verwendung der \textit{DeepSeek}-Modelle}
Das \textit{Finetuning} des kleinsten \textit{DeepSeek}-Modells lieferte Genauigkeiten von bis zu $0,866$.

Für die Verwendung der \textit{DeepSeek}-Modelle ohne \textit{Finetuning} mittels direkter Anfragen wurden Genauigkeiten bis zu $0,977$ erzielt.
Hier gilt, dass die Genauigkeit steigt, je mehr Parameter das destillierte Modell hat.
Weiterhin ist die Genauigkeit höher, wenn die Anfragen mit einem \textit{Query}-Term durchgeführt werden.
Die Ergebnisse sind in Tabelle~\ref{tab:deepseek-results} zusammengefasst.
\begin{table}
    \center
    \begin{tabular}{lcc}
        \toprule
        Modell           & Genauigkeit mit Query-Ausdruck & Genauigkeit ohne Query-Ausdruck \\
        \midrule
        DeepSeek-r1-70B  & 0,977                           &  0,930                         \\
        DeepSeek-r1-32B  & 0,966                           &  0,927                         \\
        DeepSeek-r1-8B   & 0,955                           &  0,916                         \\
        DeepSeek-r1-1.5B & 0,883                           &  0,824                         \\
        \bottomrule
    \end{tabular}
    \caption{Genauigkeit bei Verwendung der \textit{DeepSeek}-Modelle ohne \textit{Finetuning}}
    \label{tab:deepseek-results}
\end{table}


\section{Ausblick}\label{sec:ausblick}

Es gibt mehrere Ansätze, um die Vorgehensweise und die Ergebnisse der Stimmungsanalyse zu optimieren.

\subsection{\textit{Noisy Labels}}
Wie in Abschnitt \ref{subsec:trainingsdatensatz} beschrieben, führt die Art, wie der Trainingsdatensatz \textit{Sentiment140} erstellt wurde, zu \textit{Noisy Labels}.
Dies kann bei der Verwendung einfacherer Modelle zu schlechteren Klassifikationsergebnissen führen und die Anzahl der benötigten Trainingsbeispiele oder die Komplexität der Modelle erhöhen \cite{NoisyLabel2014}.
Die erzielten Genauigkeiten der klassischen Ansätze (bis zu 85\%) und die Ergebnisse von  Go et al. ($79-83\%$, vgl. Tabelle 6 in \cite{go2009twitter}) bestätigen diese Aussage.

Um den Effekt von \textit{Noisy Labels} zu verringern, können robustere Modelle oder Methoden der Datenbereinigung verwendet werden, also Datensätze entfernt oder sogenannte semi-überwachte Lernverfahren angewendet werden. Alternativ können auch tolerante Lernverfahren angewandt werden (vgl. Abschnitt 3 in \cite{NoisyLabel2014}).

Eine grundsätzlich andere Herangehensweise ist die Verwendung eines Datensatzes mit höherer Datenqualität, um \textit{Noisy Labels} zu vermeiden.

\subsection{\textit{Aspect Based Sentiment Analysis}}

Die erlangten Ergebnisse der unterschiedlichen Modelle können durch die Verwendung zusätzlicher Informationen verbessert werden, wie die Ergebnisse der Experimente mit den \textit{DeepSeek}-basierten Modellen zeigen.

Bei der \textit{Aspect Based Sentiment Analysis} werden explizite oder implizite Begriffe (\textit{aspects}) aus dem Tweet extrahiert und die Stimmung des Tweets bezüglich dieser Begriffe klassifiziert \cite{Hua_2024}.

Ein Beispiel aus \cite{Hua_2024} zeigt, dass der Satz \glqq \textit{The restaurant was expensive, but the menu was great.}\grqq{} bezüglich dem expliziten Begriff \textit{menu} positiv und bezüglich dem impliziten Begriff \textit{price} negativ klassifiziert wird.

\subsection{Große \textit{Deep Learning} Modelle}

Das \textit{Finetuning} der größeren \textit{DeepSeek}-Modelle mit dem \textit{Sentiment140}-Datensatz stieß auf Rechenkapazitätsgrenzen.

Mehr Rechenkapazitäten oder Methoden wie die \textit{Low Rank Adaptation} (LoRA) \cite{lora2021}, bei der nur eine kleine Anzahl neuer Parameter zur Feinabstimmung genutzt wird, könnten hier Abhilfe schaffen.


\section{Zusammenfassung und Fazit}\label{sec:zusammenfassung-und-fazit}

In diesem Projekt wurden verschiedene Methoden des maschinellen Lernens zur Stimmungsanalyse auf dem Sentiment140-Twitter-Datensatz evaluiert.
Drei klassische Algorithmen – \gls{lr}, \gls{svm} und \gls{nb}-Klassifikator – erreichten maximale Genauigkeiten von bis zu 85\%.
Zwei Transformer-Modelle (\emph{distilbert-base-uncased} und \emph{twitter-roberta-base-sentiment}) wurden durch \textit{Finetuning} angepasst und erzielten Genauigkeiten von 85\% bzw. 92\%.
Zusätzlich wurde versucht, die durch Modelldestillation erzeugten \emph{DeepSeek-R1}-Modelle mittels \textit{Finetuning} anzupassen.
Aufgrund der benötigten Hardware-Res\-sour\-cen war dies lediglich für die kleinste Variante (\emph{Deepseek-R1-1.5B}) möglich, die Genauigkeiten von bis zu 87\% lieferte.
Das größte verwendete \emph{DeepSeek-R1} basierte Modell erreichte die höchste Genauigkeit mit 93\% mit dem Anfrage-basiertem Ansatz.
Mit Berücksichtigung von Aspekten beim Anfrage-basiertem Ansatz wurden mit den \emph{DeepSeek-R1} Modellen Genauigkeiten bis zu 98\% erzielt.

Die Ergebnisse zeigen, dass domänenspezifische Transformer-Modelle bei der Stimmungsanalyse höhere Genauigkeiten erzielen als die analysierten klassischen \gls{ml}-Ansätze.
Obwohl die \glspl{llm} mit den meisten Parametern die höchsten Genauigkeiten erreichten, kann in der Praxis der Einsatz kleinerer Transformer-Modelle (z. B. \gls{roberta}) für Twitter-Stimmungsanalysen vorteilhafter sein – insbesondere bei begrenzten Hardware-Ressourcen oder dem Einsatz in Anwendungen mit strengen Latenzanforderungen.


% References
\newpage
\addreferences

\makestatement{5}

% ==== Appendix ====
\appendix

\section{Algorithmen}

\subsection{Datenaufbereitung}\label{subsec:appendix-data-preparation}

\textit{Hinweis}: Die Datensätze enthalten weitere Daten je Tweet, wie beispielsweise die Sentiment Klasse.
Der Umgang mit diesen Daten wurde hier aus Gründen der Übersichtlichkeit nicht aufgeführt.

Für die unterschiedlichen Parameter-Ausprägungen für die Normalisierungsverfahren und Stoppwortlisten wurden separate Datensätze erstellt.

\subsubsection{Datenaufbereitung - Algorithmus}
Die Datensätze wurden gemäß des folgenden Algorithmus erzeugt.
\begin{algorithm}
    \caption{Datenaufbereitung}
    \begin{algorithmic}[1]
        \Procedure{DatasetPreparation}{$dataset, normalizerFunction, stoppwords$}
            \Function{SanitizeTweet}{$text$}
                \State $text$ $\gets$ Entferne URLs, Nutzer, Hashtags und Sonderzeichen aus $text$
                \State \Return $text$
            \EndFunction

            \Function{NormalizeTweet}{$text, normalizerFunction, stoppwords$}
                \State $tokens$ $\gets$ Zerlege $text$ mit TweetTokenizer in Token
                \For{$i \gets 1$ \textbf{to} $|tokens|$}
                    \State $tokens$[$i$] $\gets$ $normalizerFunction$($tokens$[$i$])
                \EndFor
                \State $tokens$ $\gets$ Entferne Stoppwörter aus $tokens$ gemäß Stoppwortliste $stoppwords$
                \State $text$ $\gets$ Füge Elemente aus $tokens$ zu einem Text zusammen
                \State \Return $text$
            \EndFunction

            \For{$i \gets 1$ \textbf{to} $|dataset|$}
                \State $dataset[i]$ $\gets$ \Call{SanitizeTweet}{$dataset[i]$}
                \State $dataset[i]$ $\gets$ \Call{NormalizeTweet}{$dataset[i], normalizerFunction, stoppwords$}
            \EndFor
            \State save $dataset$
        \EndProcedure
    \end{algorithmic}
    \label{alg:data-preparation}
\end{algorithm}

\subsubsection{Datenaufbereitung - Beispiel}

Im folgenden ist ein Beispiel für die Datenaufbereitung eines Tweets dargestellt.
\begin{itemize}
    \item \textbf{Original-Tweet}: \textit{\glqq @user I love this movie! http://example.com\grqq}
    \item \textbf{Bereinigung}: \textit{\glqq I love this movie\grqq}
    \item \textbf{Tokenisierung}: \textit{\glqq I\grqq, \glqq love\grqq, \glqq this\grqq, \glqq movie\grqq}
    \item \textbf{Normalisierung}:
    \begin{itemize}
        \item \textit{Lemmatisierung}: \textit{\glqq I\grqq, \glqq love\grqq, \glqq this\grqq, \glqq movie\grqq}
        \item \textit{Stemming}: \textit{\glqq I\grqq, \glqq lov\grqq, \glqq thi\grqq, \glqq movi\grqq}
    \end{itemize}
    \item \textbf{Stoppwörter Behandlung}: Ohne Stoppwörter \glqq{}I\grqq{} und \glqq{}this\grqq{}: \textit{\glqq love\grqq, \glqq movie\grqq}
    \item \textbf{Aufbereiteter Text}: \textit{\glqq love movie\grqq}
\end{itemize}

\subsection{Training und Evaluation der Modelle}

Die Schritte für das Training und die Evaluation der Modelle sind, aufbauend auf dem durch Algorithmus \ref{alg:data-preparation} erzeugten Datensätzen wie folgt:
\begin{algorithm}
    \caption{Training und Evaluation der Modelle}
    \begin{algorithmic}[1]
        \Procedure{ModelTraining}{$dataset, vectorizer, model$}
            \State $X, y \gets$ Extrahiere Texte und Labels aus $dataset$
            \State $X \gets$ Transformiere Texte in $X$ in Vektoren mit Vektorizer $vectorizer$
            \State $X_{train}, X_{test}, y_{train}, y_{test} \gets$ Teile $X$ und $y$ in Trainings- und Testdaten
            \State $model \gets$ Trainiere $model$ auf $X_{train}$ und $y_{train}$
            \State $accuracy \gets$ Evaluiere $model$ auf $X_{test}$ und $y_{test}$
            \State \Return $accuracy$
        \EndProcedure
    \end{algorithmic}
    \label{alg:model-training}
\end{algorithm}

\section{Tabellen}

Die folgende Tabelle zeigt die einzelnen Werte für Datensatzgröße und Lernrate, welche für das \textit{Finetuning} der DL-Ansätze verwendet wurden.

\begin{table}[h]
    \center
    \begin{tabular}{lc}
        \toprule
        Parameter       & Werte                                                   \\
        \midrule
        Datensatzgröße  & 2500, 5000, 7500, 10000, 15000, 20000                   \\
        Lernrate        & $10^{-4}$, $5\cdot 10^{-5}$ , $10^{-5}$, $5\cdot 10^{-6}$, $10^{-6}$  \\
        \bottomrule
    \end{tabular}
    \caption{Parameter für das \textit{Finetuning} der \gls{bert}-Modelle}
    \label{tab:dl-params}
\end{table}

\section{Diagramme}

Im folgenden Diagramm sind für alle Modelle und Verfahren die maximalen Genauigkeiten dargestellt.

\begin{figure}[H]
    \centering
    \includegraphics[width=1.0\textwidth]{figures/alle-übersicht-genauigkeit-alle-modelle.png}
    \caption{Maximale Genauigkeit der Modelle und Verfahren}
    \label{fig:results}
\end{figure}


\end{document}
