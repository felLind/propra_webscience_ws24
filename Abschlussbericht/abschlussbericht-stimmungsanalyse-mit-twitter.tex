\documentclass[researchlab,group,]{AIGpaper}

%%%% Package Imports %%%%%%%%%%%%%%%%%%%%%%%%%%%%%%%%%%%%%%%%%%%%%%%%%%%%%%%%%%%%%%%%%
\usepackage{graphicx}					    % enhanced support for graphics
\usepackage{tabularx}				      	% more flexible tabular
\usepackage{amsfonts}					    % math fonts
\usepackage{amssymb}					    % math symbols
\usepackage{amsmath}					    % overall enhancements to math environment

%%%% optional packages
\usepackage{tikz}                           % creating graphs and other structures
\usepackage{glossaries}                     % glossaries package for glossary entries
\usepackage{soul}                           % for highlighting text (temporary usage)
\newcommand{\redhl}{\sethlcolor{red}\hl}


%%%% Author and Title Information %%%%%%%%%%%%%%%%%%%%%%%%%%%%%%%%%%%%%%%%%%%%%%%%%%%
\author{Anne Huber, Andreas Franke, Felix Lindner, Burak Özkan, Milomir Soknic}

\title{Stimmungsanalyse mit Twitter}

%%%% Glossary %%%%%%%%%%%%%%%%%%%%%%%%%%%%%%%%%%%%%%%%%%%%%%%%%%%%%%%%%%%%%%%%%%%%%%
\newacronym{svm}{SVM}{\textit{Support Vector Machine}}

%%%% Abstract %%%%%%%%%%%%%%%%%%%%%%%%%%%%%%%%%%%%%%%%%%%%%%%%%%%%%%%%%%%%%%%%%%%%%%

\germanabstract{
    \redhl{TODO: Zusammenfassung formulieren...}
}

\begin{document}

\maketitle % prints title and author information, as well as the abstract


% ===================== Beginning of the actual text section =====================
\newpage
\section{Einleitung}

\section{Aufgabenverteilung}
% Ein Abschnitt pro Teammitglied
% Kurze Übersicht, was die Person innerhalb des Praktikums beigetragen hat, insbesondere
% entwickelter Code, Beitrag zum Abschlussbericht, organisatorischer Beitrag, Beitrag zur
% Abschlusspräsentation, etc.

\subsection{Anne Huber}
\subsection{Andreas Franke}
\subsection{Felix Lindner}
\subsection{Burak Özkan}
\subsection{Milomir Soknic}

\section{Teaminterne Organisation}
% Wie wurde innerhalb des Teams kommuniziert?
% Welche Programmiersprache? Warum?
% Welche Tools/Techniken wurden verwendet?
% etc.

\section{Datensätze und Problemstellung}
% Kurze Beschreibung der Datensätze
% Welches Problem soll mithilfe der Datensätze gelöst werden?

\subsection{Datensätze}

Es wurden zwei Datensätze verwendet, die in den beiden folgenden Unterabschnitten beschrieben werden.

\redhl{TODO: Lizenzinformationen hinzufügen}

\subsubsection{Trainingsdatensatz}
Der verwendete Datensatz \glqq Sentiment140\grqq{} wurde von Go et al. \cite{go2009twitter} erzeugt und enthält 1.600.000 modifizierte Tweets.
Die Tweets wurden im Zeitraum April 2009 bis Juni 2009 erstellt.
Jedem Tweet im Datensatz ist eine Stimmungsklasse zugeordnet, die angibt, ob der Tweet eine positive oder negative Stimmung ausdrückt.

Die Tweets wurden mit Hilfe der Twitter-API gesammelt, indem nach Tweets gesucht wurde, die bestimmte Emoticons mit positiver oder negativer Bedeutung enthalten.
Die von Go et al. verwendeten Emoticons sind in Tabelle 3 \cite[S. 4]{go2009twitter} aufgeführt.
Anhand der verwendeten Emoticons der Suchanfrage wurden die Tweets in Klassen mit positiver und negativer Stimmung eingeteilt.
Tweets, die sowohl positive als auch negative Emoticons enthalten, sind nicht im Datensatz enthalten.
Go et al. \cite{go2009twitter} weisen darauf hin, dass die Zuordnung der Stimmungsklassen nicht fehlerfrei ist\footnote{
    Als Beispiel wird von Go et al. \cite{go2009twitter} ein Tweet mit dem Text \glqq @BATMANNN :( i love chutney.......\grqq{} genannt, der fälschlicherweise als negativ klassifiziert wird, dessen Inhalt aber eher als positiv angesehn würde.
} und dass die Stimmungsklassen als Noisy Labels verwendet wurden.
Für beide Stimmungsklassen enthält der Datensatz 800.000 Einträge.

Die Tweet Texte im Datensatz wurden so angepasst, dass die Emoticons die zur Einteilung verwendet wurden, entfernt wurden.
Weiterhin enthält der Datensatz keine Re-Tweets\footnote{
    Zum Zeitpunkt der Datenerstellung gab es keine automatisierte Möglichkeit Tweets zu teilen.
    Um Tweets zu teilen wurde deshalb dem Tweet Text die Kennzeichnung \textit{RT} und der Benutzername des ursprünglichen Autors vorangestellt.
} und keine Duplikate.

\subsubsection{Testdatensatz}

Von Go et al. \cite{go2009twitter} wurde ein Testdatensatz mit 498 Tweets erstellt, der 177 Tweets mit positiver Stimmung und 189 Tweets mit negativer Stimmung enthält.
Die restlichen Tweets sind als neutral klassifiziert.
Die Tweets wurden manuell ausgewählt und enthielten nicht in jedem Fall Emoticons.

\subsection{Problemstellung}

Stimmungsanalyse befasst sich mit der automatischen Erkennung von Stimmungen und deren Polarität in Texten \cite{giachanou2016like, jianqiang2017comparison, zimbra2018state}.

Die Stimmungsanalyse von Tweets ist eine spezielle Form der Stimmungsanalyse.
Nach Zimbra et al. \cite{zimbra2018state} wird im Rahmen der Stimmungsanalyse für Twitter häufig versucht die Tweets in zwei oder mehr Klassen einzuteilen (wie in den ausgewählten Trainings- und Testdatensätzen).

Die Analyse der Stimmung von Tweets wird als besonders herausfordernd angesehen \cite{agarwal2011sentiment, giachanou2016like, zimbra2018state}.
Giachanou und Crestani \cite{giachanou2016like} nennen neben der Längenbeschränkung von Tweets auf 140 Zeichen\footnote{
    Seit November 2017 sind 280 Unicode-Zeichen pro Tweet erlaubt.
} insbesondere die informelle Art von Tweets als Herausforderung in Bezug auf Stimmungsanalysen.
Agarwal et al. \cite{agarwal2011sentiment} und Zimbra et al. \cite{zimbra2018state} weisen darauf hin, dass aufgrund der Längenbeschränkung besonders häufig Abkürzungen, Emoticons und andere Zeichen mit spezieller Bedeutung oder Umgangssprache in Tweets verwendet werden.
\newline


Das Ziel dieser Arbeit ist es, bestehende Methoden zur automatischen Stimmungsanalyse von Tweets anzuwenden und zu evaluieren.
Dabei sollen sowohl klassische maschinelle Lernverfahren als auch moderne Deep-Learning-Ansätze untersucht werden.

Die zentrale Problemstellung lautet: Wie effektiv sind verschiedene maschinelle Lernverfahren bei der Stimmungsanalyse von Tweets?

Insbesondere soll untersucht werden, welche Ansätze die besten Ergebnisse in Bezug auf Genauigkeit der Klassifikation liefern.
Darüber hinaus sollen die Herausforderungen und Limitationen der Stimmungsanalyse von Tweets identifiziert und diskutiert werden.


\section{Ansätze}
% Welche klassischen, Deep Learning -, und eigenen Ansätze wurde verwendet?
% Kurze Beschreibung neuer Techniken und Ideen

\subsection{Klassische Ansätze}

Als klassische Ansätze zur Stimmungsanalyse von Tweets wurden folgende überwachten Lernverfahren ausgewählt, weil diese besonders häufig in der Literatur zu finden sind \cite{wankhade2022survey, medhat2014sentiment, zimbra2018state}:

\begin{itemize}
    \item Logistische Regression
    \item \gls{svm}
    \item Naiver Bayes Klassifikator
\end{itemize}

\redhl{TODO: Ggf. Anmerkung ergänzen, dass zwei weitere klassische Ansätze evaluiert wurden.}

\subsubsection{Logistische Regression}

Die logistische Regression ist ein Verfahren zur Klassifikation, das auf der \Sigmoid-Funktion basiert. Diese Funktion ist definiert als:

\begin{equation*}
    \sigma(z) = \frac{1}{1 + e^{-z}}
\end{equation*}

wodurch Werte aus dem gesamten Zahlenraum auf das Intervall $(0,1)$ abgebildet werden. Die Entscheidungsregel für einen Datenpunkt $x$ ergibt sich durch:

\begin{equation*}
    h_{\theta}(x) = \sigma(\theta \cdot x + b), \text{ wobei } \theta \in \mathbb{R}^{n}, b \in \mathbb{R}
\end{equation*}

Ein Datenpunkt wird dabei der Klasse 1 zugeordnet, wenn $h_{\theta}(x) \geq 0.5$, ansonsten der Klasse 0:

\begin{equation*}
    clf_{\theta, b}(x) =
    \begin{cases}
        1, & \text{wenn } h_{\theta}(x) \geq 0.5 \\
        0, & \text{sonst}
    \end{cases}
\end{equation*}

Zur Bestimmung der Parameter $\theta$ und $b$ wird die logistische Verlustfunktion minimiert:

\begin{equation*}
    \min_{\theta, b} \frac{1}{m} \sum_{i=1}^{m} \left[-y_i \log h_{\theta}(x_i) - (1 - y_i) \log(1 - h_{\theta}(x_i))\right]
\end{equation*}

Dieses Optimierungsproblem ist konvex und wird mithilfe des \textit{Gradient Descent} gelöst. Dabei werden die Parameter iterativ durch:

\begin{equation*}
    \theta \leftarrow \theta - \alpha \nabla_{\theta} L
\end{equation*}

aktualisiert, wobei $\alpha > 0$ die Lernrate ist. Eine Regularisierung kann durch einen zusätzlichen Term $\lambda \lVert\theta\rVert^2$ eingeführt werden, um Überanpassung zu vermeiden.

\subsubsection{\textit{Support Vector Machine}}

Die \gls{svm} wird oftmals zur Klassifikation verwendet.
Die Klassifikation erfolgt dabei durch die Bestimmung einer Hyperebene, die die Daten in zwei Klassen trennt.
Eine Hyperebene in $\mathbb{R}^{n}$ ist definiert als die Menge aller Punkte $x\in\mathbb{R}^n$ für die gilt:
\begin{equation*}
    \theta \cdot x - b = 0, \text{ wobei } \theta \in \mathbb{R}^{n}, b\in\mathbb{R}
\end{equation*}

Für linear nicht separierbare Datensätze, lässt sich keine Hyperebene finden, die die Daten perfekt trennt.
Stattdessen wird versucht eine Hyperebene zu bestimmen, die die Daten unter Berücksichtigung der \textit{Hinge}-Fehlerfunktion\footnote{
Die \textit{Hinge}-Fehlerfunktion ist definiert als:
\begin{equation*}
    L^{hinge}(D, \theta, b) = \sum_{1}^{m}\max\lbrace0, 1 - y_i(\theta \cdot x_i - b)\rbrace
\end{equation*}
} möglichst gut trennt.
Die optimalen Parameter zur Bestimmung der Hyperebene können durch folgendes Minimierungsproblem, für einen Regularisierungsparameter $C\geq0$, bestimmt werden:
\begin{equation*}
    \min_{w, b} \lvert\lvert \theta \rvert\rvert + C \frac{1}{m}\sum_{i=1}^{m} \max\lbrace0, 1 - y_i(\theta \cdot x_i - b)\rbrace
\end{equation*}

Die binäre Klassifikation von Datenpunkten erfolgt dann durch die Bestimmung der Klasse des Datenpunktes $x$ durch:
\begin{equation*}
        clf_{\theta, b}(x) =
        \begin{cases}
        1, & \text{wenn } \theta \cdot x - b \geq 0 \\
        -1, & \text{sonst}
    \end{cases}
\end{equation*}


\subsubsection{Naiver Bayes Klassifikator}
Der \textit{naive Bayes-Klassifikator}, benannt nach dem englischen Mathematiker Thomas Bayes, ist ein maschinelles Lernverfahren, das aufgrund seiner Einfachheit und Effizienz häufig für Klassifikationsprobleme eingesetzt wird \cite{wankhade2022survey, medhat2014sentiment, zimbra2018state}.

Das Ziel ist es, für einen \textit{Trainingsdatensatz} $D$ eine optimale \textit{Hypothese} $h$ zu finden. Das Verfahren basiert auf dem \textit{Bayes-Theorem}, welches uns ermöglicht, uns der gesuchten optimalen Wahrscheinlichkeit $P(h|D)$ anzunähern. Dabei ist $P(h|D)$ die Wahrscheinlichkeit von $h$, gegeben der Beobachtung $D$:

\begin{equation*}
    P(h|D) = \frac{P(D|h)P(h)}{P(D)}
\end{equation*}

Wir suchen also eine Hypothese $h^*$ die den Wert $P(h|D)$ maximiert:
\begin{equation*}
    h^* = \arg\max_{h} P(h|D)
\end{equation*}

Die Grundidee des naiven Bayes-Klassifikators ist die Annahme, dass die einzelnen Merkmale unabhängig voneinander sind.

Sei $D = (x^{(1)}, y^{(1)}), \dots, (x^{(m)}, y^{(m)})$ ein Datensatz, $c$ eine Klasse im Merkmalsraum $Z$  und $x=(x_1, \dots,x_n)\in Z_1 \times \dots \times Z_n$ ein neuer Datenpunkt. Dann ist der Naive-Bayes-Klassifikator
\begin{equation*}
    clf_D^{NaiveBayes}(x) = \arg \max_{c\in Z} P(c|D)P(x_1|c,D)P(x_2|c,D) \dots P(x_n|c,D)
\end{equation*}
mit
\begin{equation*}
    P(c|D) = \frac{|\{(z,c)\in D\}|}{|D|}
\end{equation*}
\begin{equation*}
    P(x_i|c,D) = \frac{|\{(z^\prime,c)\in D| z^\prime = (z_1, \dots, z_n), z_i=x_i \}|}{|\{(z,c)\in D\}|} \quad \text{für} \: i = 1, \dots, n 
\end{equation*}

Trotz der naiven Unabhängigkeitsannahme führt dieses Verfahren in der Praxis für eine Vielzahl von Anwendungsfällen zu guten Ergebnissen \cite{hand2001idiot}. Der naive Bayes-Klassifikator wird daher neben der Sentimentanalyse häufig auch für andere Textklassifikationsaufgaben eingesetzt. Eine typische Anwendung ist z.B. die Spam-Filterung \cite{sahami1998bayesian}.

\subsection{Deep Learning Ansätze}

\section{Experimente}
% Wie ist der Experimentaufbau, welche Evaluationsmetriken betrachten Sie?
% Beschreibung und Interpretation der Ergebnisse

\subsection{Experimentaufbau}

\subsection{Klassische Ansätze}

\subsubsection{\gls{svm}}

Als Modell-Parameter wurden die Standardwerte für die \textit{LinearSVC}-Implementierung in der Python-Bibliothek \textit{scikit-learn} verwendet.
Die Standardwerte sind:

\begin{itemize}
    \item $C = 1.0$
    \item Fehlerfunktion: quadrierte Hinge-Fehlerfunktion
\end{itemize}

\subsection{Deep Learning Ansätze}

\subsection{Evalutionsmetriken}

\subsection{Ergebnisse}

\section{Ausblick}
% Was wurde nicht geschafft bzw. hat nicht funktioniert? Warum?
% Was kann noch verbessert werden? Wie?

\section{Zusammenfassung und Fazit}

In diesem Projekt wurden verschiedene Methoden des maschinellen Lernens zur Stimmungsanalyse auf dem Sentiment140-Twitter-Datensatz evaluiert. 
Drei klassische Algorithmen – logistische Regression, \gls{svm} und Naive Bayes-Klassifikator – erreichten maximale Genauigkeiten von bis zu 86\%. 
Zwei Transformer-Modelle (\emph{distilbert-base-uncased} und \emph{twitter-roberta-base-sentiment}) wurden durch Fine-Tuning angepasst und erzielten Genauigkeiten von 85\% bzw. 92\%. 
Zusätzlich wurde versucht, das \emph{Deepseek-LLM} mittels Fine-Tuning anzupassen, was jedoch aufgrund begrenzter Hardware-Ressourcen nur für die kleinste Variante (\emph{Deepseek-R1-Distill-Qwen-1.5B}) möglich war. 
Erwartungsgemäß erreichte das größte LLM (\emph{Deepseek-R1-70B}) im Zero-Shot-Ansatz (ohne anwendungsspezifisches Training) die beste Genauigkeit von bis zu 98\,\%.

Klassische Methoden liefern respektable Ergebnisse und Metriken, die mit denen des \emph{distilbert-base-uncased}-Modells vergleichbar sind. 
Die Ergebnisse unterstreichen jedoch die Überlegenheit domänenspezifischer Transformer-Modelle gegenüber allgemeineren Modellen. 
Obwohl das LLM die höchste Genauigkeit erreichte, kann in der Praxis der Einsatz ressourceneffizienter vortrainierter Transformer-Modelle (z. B. \emph{twitter-roberta}) für Twitter-Stimmungsanalysen oft sinnvoller sein – insbesondere bei begrenzten Hardware-Ressourcen.

% References
\newpage
\addreferences

\makestatement{5}

\end{document}
