\documentclass[researchlab,group,]{AIGpaper}

%%%% Package Imports %%%%%%%%%%%%%%%%%%%%%%%%%%%%%%%%%%%%%%%%%%%%%%%%%%%%%%%%%%%%%%%%%
\usepackage{graphicx}					    % enhanced support for graphics
\usepackage{tabularx}				      	% more flexible tabular
\usepackage{amsfonts}					    % math fonts
\usepackage{amssymb}					    % math symbols
\usepackage{amsmath}					    % overall enhancements to math environment
\usepackage{multirow}					    % multirow for tables
\usepackage{booktabs}					    % improved table design

%%%% optional packages
\usepackage{tikz}                           % creating graphs and other structures
\usepackage{glossaries}                     % glossaries package for glossary entries
\usepackage{soul}                           % for highlighting text (temporary usage)
\usepackage{algorithm}                      % for creating pseudo-code
\usepackage{algpseudocode}                  % for creating pseudo-code
\usepackage{multicol}                       % for creating multi-column environments
\usepackage{enumitem}                       % for reducing space in lists
\usepackage{hyphenat}                       % for hyphenation
\usepackage[hidelinks]{hyperref} % or \usepackage{hyperref}


\setlist{noitemsep}

\newcommand{\redhl}{\sethlcolor{red}\hl}
\newcommand{\greyhl}{\sethlcolor{lightgray}\hl}


%%%% Author and Title Information %%%%%%%%%%%%%%%%%%%%%%%%%%%%%%%%%%%%%%%%%%%%%%%%%%%
\author{Anne Huber, Andreas Franke, Felix Lindner, Burak Özkan, Milomir Soknic}

\title{Stimmungsanalyse mit Twitter}

%%%% Glossary %%%%%%%%%%%%%%%%%%%%%%%%%%%%%%%%%%%%%%%%%%%%%%%%%%%%%%%%%%%%%%%%%%%%%%

\newacronym{bert}{BERT}{\textit{Bidirectional Encoder Representations from Transformers}}
\newacronym{cnn}{CNN}{\textit{Convolutional Neural Network}}
\newacronym{cl}{CL}{Computerlinguistik}
\newacronym{dl}{DL}{\textit{Deep Learning}}
\newacronym{knn}{KNN}{\textit{k-Nearest Neighbors}}
\newacronym{llm}{LLM}{\textit{Large Language Model}}
\newacronym{lr}{LR}{Logistische Regression}
\newacronym{lstm}{LSTM}{\textit{Long Short-Term Memory}}
\newacronym{ml}{ML}{Maschinelles Lernen}
\newacronym{nb}{NB}{Naive Bayes}
\newacronym{nlp}{NLP}{\textit{Natural Language Processing}}
\newacronym{nltk}{NLTK}{\textit{Natural Language Toolkit}}
\newacronym{roberta}{RoBERTa}{\textit{Robustly optimized BERT approach}}
\newacronym{sgd}{SGD}{\textit{Stochastic Gradient Descent}}
\newacronym{svm}{SVM}{\textit{Support Vector Machine}}
\newacronym{tfidf}{TF-IDF}{\textit{Term Frequency-Inverse Document Frequency}}

%%%% Abstract %%%%%%%%%%%%%%%%%%%%%%%%%%%%%%%%%%%%%%%%%%%%%%%%%%%%%%%%%%%%%%%%%%%%%%

\germanabstract{
    Dieser Bericht beschreibt die Durchführung und Ergebnisse eines Projekts zur Stimmungsanalyse von Tweets.
    Ziel des Projekts war es, die Effektivität verschiedener maschineller Lernverfahren, einschließlich klassischer Methoden und Deep-Learning-Ansätze, bei der Klassifizierung der Stimmung von Tweets zu untersuchen.
    Dazu wurden verschiedene Vorverarbeitungsschritte und Vektorisierungsverfahren für klassische Methoden angwendet und evaluiert.
    Für die Deep-Learning Ansätze wurden \textit{Finetuning} Ansätze für \textit{BERT}-basierte Modelle und \textit{DeepSeek}-R1 basierte Modelle verwendet. Des Weiteren wurde ein aspekt-basierter Ansatz mit den \textit{DeepSeek}-R1 basierten Modellen untersucht.
    Die Ergebnisse zeigen, dass Deep-Learning-Modelle, insbesondere fein-abgestimmte \textit{BERT}-basierte Modelle sowie aktuelle \textit{LLMs}, eine höhere Genauigkeit bei der Stimmungsanalyse erzielen als klassische maschinelle Lernverfahren.
}


\begin{document}

\maketitle % prints title and author information, as well as the abstract


% ===================== Beginning of the actual text section =====================

\section{Einleitung}

Twitter zählt zu den bedeutendsten sozialen Netzwerken, auf dem täglich Millionen von Nutzenden ihre Meinungen zu unterschiedlichen Themen äußern.
Aufgrund der enormen Menge an Tweets bietet die Plattform wertvolle Einblicke in die öffentliche Meinung und gesellschaftliche Diskurse \cite{pak2010twitter}.
Die Stimmungsanalyse, ein Teilgebiet der natürlichen Sprachverarbeitung (NLP), ermöglicht es, Emotionen und Meinungen automatisiert zu erkennen und zu kategorisieren.
Insbesondere im Bereich der Web Science trägt diese Methode dazu bei, Dynamiken in sozio-technischen Systemen besser zu verstehen.

Im Rahmen dieses Projekts werden verschiedene Verfahren des maschinellen Lernens auf einen Twitter-Datensatz angewendet, um die Stimmung der Nutzer zu analysieren.
Als Datengrundlage dient der \glqq Sentiment140\grqq{}-Datensatz von \textit{HuggingFace}\footnote{https://huggingface.co/datasets/stanfordnlp/sentiment140}, der eine große Menge annotierter Tweets enthält \cite{go2009twitter}.
Ziel der Analysen ist es, unterschiedliche Ansätze, einschließlich Deep-Learning-Methoden, hinsichtlich ihrer Fähigkeit zur Stimmungsanalyse zu evaluieren.

Im Bericht wird zunächst die Aufgabenverteilung im Team sowie die genutzten Tools und Methoden beschrieben.
Anschließend werden die Schritte der Datenaufbereitung, die eingesetzten Modelle und die durchgeführten Experimente erläutert.
Abschließend erfolgt eine Diskussion der Ergebnisse, gefolgt von einem Ausblick auf mögliche Optimierungen und weiterführende Forschungsansätze.


\section{Aufgabenverteilung}\label{sec:aufgabenverteilung}
% Ein Abschnitt pro Teammitglied
% Kurze Übersicht, was die Person innerhalb des Praktikums beigetragen hat, insbesondere
% entwickelter Code, Beitrag zum Abschlussbericht, organisatorischer Beitrag, Beitrag zur
% Abschlusspräsentation, etc.

\begin{multicols}{2}
[
Die Themen Literaturrecherche und Ideenfindung, Überprüfung der Datenqualität des gegebenen Datensatzes, Suche nach einem neuen Datensatz und die explorative Datenanalyse des \textit{Sentiment140} Datensatzes wurden von allen Projektmitgliedern bearbeitet.
Darüber hinaus wurden von allen Projektmitgliedern ein klassischen Verfahren angewandt und unterschiedliche Kombinationen von Vorverarbeitungsschritten in Bezug auf die Genauigkeit getestet.
Außerdem wurde die Zwischen- und Abschlusspräsentation nach der initialen Erstellung gemeinsam überarbeitet.
Der Abschnitte des Abschlussberichts wurden anfänglich von einzelnen Mitgliedern geschrieben und anschließend gemeinsam überarbeitet.
Im folgenden sind die Aufgaben aufgezählt, die nicht von allen Mitgliedern durchgeführt wurden.
]

\subsection{Anne Huber}
\begin{itemize}
    \item Kommunikation mit den Projektbetreuern
    \item Protokollerstellung der Treffen
    \item Halten der Zwischenpräsentation
    \item Klassische Verfahren: \gls{knn}
    \item Abschlussbericht:
    \begin{itemize}
        \item \ref{sec:aufgabenverteilung} Aufgabenverteilung
        \item \ref{sec:teaminterneorganisation} Teaminterne Organisation
        \item \ref{sec:ausblick} Ausblick
    \end{itemize}
\end{itemize}


\subsection{Andreas Franke}
\begin{itemize}
    \item Klassische Verfahren: SVM
    \item Implementierung des Frameworks für die klassischen Verfahren
    \item Implementierung \textit{Finetuning}-Logik für die BERT-basierten Ansätze
    \item Implementierung der Prompting-Logik für die \textit{DeepSeek} Modelle
    \item Durchführung Abläufe für die BERT-basierten Ansätze
    \item Durchführung der \textit{Zero-Shot}-Ansätze (\textit{DeepSeek} R1 1.5B, 8B, 32B)
    \item Abschlussbericht:
    \begin{itemize}
        \item \ref{sec:datensätzeundproblemstellung} Datensätze und Problemstellung
        \item \ref{subsec:klassische-ansaetze} Klassische Ansätze
        \item \ref{subsubsec:experimente-klassische-ansaetze} Klassische Ansätze
        \item \ref{subsec:modell-parameter-und-evaluationsmetriken} Modell-Parameter und Evaluationsmetriken
        \item \ref{subsubsec:ergebnisse-klassische-ansaetze} Klassische Ansätze
    \end{itemize}
\end{itemize}

\subsection{Felix Lindner}
\begin{itemize}
    \item Klassische Verfahren: Entscheidungsbäume
    \item Erweiterung des Frameworks für die klassischen Verfahren
    \item Implementierung \textit{Finetuning}-Logik für die BERT-basierten Ansätze
    \item Durchführung Abläufe für die BERT-basierten Ansätze
    \item Durchführung des \textit{Zero-Shot}-Ansatzes (\textit{DeepSeek} R1 70B)
    \item Halten der Abschlusspräsentation
    \item Abschlussbericht:
    \begin{itemize}
        \item \ref{subsec:deep-learning-ansaetze} \textit{Deep Learning} Ansätze
        \item \ref{subsubsec:experimente-deep-learning-ansaetze} \textit{Deep Learning} Ansätze
        \item \ref{subsubsec:ergebnisse-deep-learning-ansaetze} \textit{Deep Learning} Ansätze
    \end{itemize}
\end{itemize}

\subsection{Burak Özkan}
\begin{itemize}
    \item Klassische Verfahren: Naiver Bayes Klassifikator
    \item Erstellung der Abschlusspräsentation
    \item Abschlussbericht:
    \begin{itemize}
        \item \ref{sec:zusammenfassung-und-fazit} Zusammenfassung und Fazit
    \end{itemize}
\end{itemize}

\subsection{Milomir Soknic}
\begin{itemize}
    \item Klassische Verfahren: Logistische Regression
    \item Erstellung der Zwischenpräsentation
    \item Erstellung der Abschlusspräsentation
    \item Abschlussbericht:
    \begin{itemize}
        \item \ref{sec:einleitung} Einleitung
    \end{itemize}
\end{itemize}

\end{multicols}


\section{Teaminterne Organisation}
% Wie wurde innerhalb des Teams kommuniziert?
% Welche Programmiersprache? Warum?
% Welche Tools/Techniken wurden verwendet?
% etc.


\section{Datensätze und Problemstellung}
% Kurze Beschreibung der Datensätze
% Welches Problem soll mithilfe der Datensätze gelöst werden?

\subsection{Datensätze}

Es wurden zwei Datensätze verwendet, die in den beiden folgenden Unterabschnitten beschrieben werden.

\redhl{TODO: Lizenzinformationen hinzufügen}

\subsubsection{Trainingsdatensatz}\label{traindata}
Der verwendete Datensatz \glqq Sentiment140\grqq{} wurde von Go et al. \cite{go2009twitter} erzeugt und enthält 1.600.000 modifizierte Tweets.
Die Tweets wurden im Zeitraum April 2009 bis Juni 2009 erstellt.
Jedem Tweet im Datensatz ist eine Stimmungsklasse zugeordnet, die angibt, ob der Tweet eine positive oder negative Stimmung ausdrückt.

Die Tweets wurden mit Hilfe der Twitter-API gesammelt, indem nach Tweets gesucht wurde, die bestimmte Emoticons mit positiver oder negativer Bedeutung enthalten.
% Zitierung der Tabelle anpassen - ggf. Tabelle in Appendix einfügen.
Die von Go et al. verwendeten Emoticons sind in Tabelle 3 \cite[S. 4]{go2009twitter} aufgeführt.
Anhand der verwendeten Emoticons der Suchanfrage wurden die Tweets in Klassen mit positiver und negativer Stimmung eingeteilt.
Tweets, die sowohl positive als auch negative Emoticons enthalten, sind nicht im Datensatz enthalten.
Go et al. \cite{go2009twitter} weisen darauf hin, dass die Zuordnung der Stimmungsklassen nicht fehlerfrei ist\footnote{
    Als Beispiel wird von Go et al. \cite{go2009twitter} ein Tweet mit dem Text \glqq @BATMANNN :( i love chutney.......\grqq{} genannt, der fälschlicherweise als negativ klassifiziert wird, dessen Inhalt aber eher als positiv angesehn würde.
} und dass die Stimmungsklassen als Noisy Labels verwendet wurden.
Für beide Stimmungsklassen enthält der Datensatz 800.000 Einträge.

Die Tweet Texte im Datensatz wurden so angepasst, dass die Emoticons die zur Einteilung verwendet wurden, entfernt wurden.
Weiterhin enthält der Datensatz keine Re-Tweets\footnote{
    Zum Zeitpunkt der Datenerstellung gab es keine automatisierte Möglichkeit Tweets zu teilen.
    Um Tweets zu teilen wurde deshalb dem Tweet Text die Kennzeichnung \textit{RT} und der Benutzername des ursprünglichen Autors vorangestellt.
} und keine Duplikate.

\subsubsection{Testdatensatz}

% Ergänzung um Query-Terms
Von Go et al. \cite{go2009twitter} wurde ein Testdatensatz mit 498 Tweets erstellt, der 177 Tweets mit positiver Stimmung und 189 Tweets mit negativer Stimmung enthält.
Die restlichen Tweets sind als neutral klassifiziert.
Die Tweets wurden manuell ausgewählt und mit Sentiment-Klassen versehen.
Die Tweets enthielten nicht in jedem Fall Emoticons.

\subsection{Problemstellung}

% Folgenden Satz in Abschnitt zur Einleitung einfügen
Stimmungsanalyse befasst sich mit der automatischen Erkennung von Stimmungen und deren Polarität in Texten \cite{giachanou2016like, jianqiang2017comparison, zimbra2018state}.

Die Stimmungsanalyse von Tweets ist eine spezielle Form der Stimmungsanalyse.
Nach Zimbra et al. \cite{zimbra2018state} wird im Rahmen der Stimmungsanalyse für Twitter häufig versucht die Tweets in zwei oder mehr Klassen einzuteilen (wie in den ausgewählten Trainings- und Testdatensätzen).

Die Analyse der Stimmung von Tweets wird als besonders herausfordernd angesehen \cite{agarwal2011sentiment, giachanou2016like, zimbra2018state}.
Giachanou und Crestani \cite{giachanou2016like} nennen neben der Längenbeschränkung von Tweets auf 140 Zeichen\footnote{
    Seit November 2017 sind 280 Unicode-Zeichen pro Tweet erlaubt.
} insbesondere die informelle Art von Tweets als Herausforderung in Bezug auf Stimmungsanalysen.
Agarwal et al. \cite{agarwal2011sentiment} und Zimbra et al. \cite{zimbra2018state} weisen darauf hin, dass aufgrund der Längenbeschränkung besonders häufig Abkürzungen, Emoticons und andere Zeichen mit spezieller Bedeutung oder Umgangssprache in Tweets verwendet werden.
\newline

Das Ziel dieser Arbeit ist es, bestehende Methoden zur automatischen Stimmungsanalyse von Tweets anzuwenden und zu evaluieren.
Dabei sollen sowohl klassische maschinelle Lernverfahren als auch moderne Deep-Learning-Ansätze untersucht werden.

% Problemstellung ggf. um Unterschiede zwischen klassichen und deep learning Verfahren erweitern
Die zentrale Problemstellung lautet: Wie effektiv sind verschiedene maschinelle Lernverfahren bei der Stimmungsanalyse von Tweets?

Insbesondere soll untersucht werden, welche Ansätze die besten Ergebnisse in Bezug auf die Genauigkeit der Klassifikation liefern.
Darüber hinaus sollen die Herausforderungen und Limitationen der Stimmungsanalyse von Tweets identifiziert und diskutiert werden.


\section{Ansätze}
% Welche klassischen, Deep Learning -, und eigenen Ansätze wurde verwendet?
% Kurze Beschreibung neuer Techniken und Ideen

In diesem Abschnitt werden die Ansätze zur Stimmungsanalyse von Tweets beschrieben, die wir verwendet haben.
In den Unterabschnitten \ref{subsec:klassische-ansaetze} und \ref{subsec:deep-learning-ansaetze} werden die verwendeten klassischen Ansätze und \gls{dl} basierten Ansätze beschrieben.

\subsection{Klassische Ansätze}\label{subsec:klassische-ansaetze}

Als klassische Ansätze zur Stimmungsanalyse von Tweets wurden folgende überwachten Lernverfahren ausgewählt, weil diese besonders häufig in der Literatur zu finden sind \cite{wankhade2022survey, medhat2014sentiment, zimbra2018state}:

\begin{itemize}
    \item Logistische Regression
    \item \gls{svm}
    \item Naiver Bayes Klassifikator
\end{itemize}

Die Verfahren \gls{knn} und Entscheidungsbäume wurden verworfen, weil diese auf dem von uns verwendeten Datensatz schlechtere Ergebnisse erzielten und darüber hinaus längere Laufzeiten benötigten.

In den folgenden Absätzen sind die drei ausgewählten klassischen Verfahren kurz beschrieben.

\subsubsection{Logistische Regression}

Die logistische Regression ist ein Verfahren zur Klassifikation, das auf der Sigmoid-Funktion basiert. Diese Funktion ist definiert als:

\begin{equation*}
    \sigma(z) = \frac{1}{1 + e^{-z}}
\end{equation*}

wodurch Werte aus dem gesamten Zahlenraum auf das Intervall $(0,1)$ abgebildet werden. Die Entscheidungsregel für einen Datenpunkt $x$ ergibt sich durch:

\begin{equation*}
    h_{\theta}(x) = \sigma(\theta^T x + b), \text{ wobei } \theta \in \mathbb{R}^{n}, b \in \mathbb{R}
\end{equation*}

Ein Datenpunkt wird dabei der Klasse 1 zugeordnet, wenn $h_{\theta}(x) \geq 0.5$, ansonsten der Klasse 0:

\begin{equation*}
    clf_{\theta, b}(x) =
    \begin{cases}
        1, & \text{wenn } h_{\theta}(x) \geq 0.5 \\
        0, & \text{sonst}
    \end{cases}
\end{equation*}

Zur Bestimmung der Parameter $\theta$ und $b$ wird die logistische Verlustfunktion minimiert:

\begin{equation*}
    \min_{\theta, b} \frac{1}{m} \sum_{i=1}^{m} \left[-y_i \log h_{\theta}(x_i) - (1 - y_i) \log(1 - h_{\theta}(x_i))\right]
\end{equation*}

Dieses Optimierungsproblem ist konvex und kann mithilfe des \textit{Gradient Descent} Verfahren gelöst werden (siehe beispielsweise \cite{jm3}).

\subsubsection{\textit{Support Vector Machine}}

Die \gls{svm} wird oftmals zur Klassifikation verwendet \cite{medhat2014sentiment, wankhade2022survey}.
Die binäre Klassifikation erfolgt dabei durch die Bestimmung einer Hyperebene, die die Daten in zwei Klassen trennt.
Eine Hyperebene in $\mathbb{R}^{n}$ ist definiert als die Menge aller Punkte $x\in\mathbb{R}^n$ für die gilt:
\begin{equation*}
    \theta^T x - b = 0, \text{ wobei } \theta \in \mathbb{R}^{n}, b\in\mathbb{R}
\end{equation*}

Für linear nicht separierbare Datensätze, lässt sich keine Hyperebene finden, die die Daten perfekt trennt.
Stattdessen wird versucht eine Hyperebene zu bestimmen, die die Daten unter Berücksichtigung der \textit{Hinge}-Fehlerfunktion\footnote{
    Die \textit{Hinge}-Fehlerfunktion ist definiert als:
    \begin{equation*}
        L^{hinge}(D, \theta, b) = \sum_{1}^{m}\max\lbrace0, 1 - y_i(\theta \cdot x_i - b)\rbrace
    \end{equation*}
} möglichst gut trennt.
Die optimalen Parameter zur Bestimmung der Hyperebene können durch folgendes Minimierungsproblem, für einen Regularisierungsparameter $C\geq0$, bestimmt werden:
\begin{equation*}
    \min_{w, b} \lvert\lvert \theta \rvert\rvert + C \frac{1}{m}\sum_{i=1}^{m} \max\lbrace0, 1 - y_i(\theta \cdot x_i - b)\rbrace
\end{equation*}

Die binäre Klassifikation von Datenpunkten erfolgt dann durch die Bestimmung der Klasse des Datenpunktes $x$ durch:
\begin{equation*}
    clf_{\theta, b}(x) =
    \begin{cases}
        1, & \text{wenn } \theta^T x - b \geq 0 \\
        -1, & \text{sonst}
    \end{cases}
\end{equation*}

\subsubsection{Naiver Bayes Klassifikator}
Der \textit{naive Bayes-Klassifikator}, benannt nach dem englischen Mathematiker Thomas Bayes, ist ein maschinelles Lernverfahren, das aufgrund seiner Einfachheit und Effizienz häufig für Klassifikationsprobleme eingesetzt wird \cite{wankhade2022survey, medhat2014sentiment, zimbra2018state}.

Das Ziel ist es, für einen \textit{Trainingsdatensatz} $D$ eine optimale \textit{Hypothese} $h$ zu finden.
Das Verfahren basiert auf dem \textit{Bayes-Theorem}, welches uns ermöglicht, uns der gesuchten optimalen Wahrscheinlichkeit $P(h|D)$ anzunähern.
Dabei ist $P(h|D)$ die Wahrscheinlichkeit von $h$, gegeben der Beobachtung $D$:

\begin{equation*}
    P(h|D) = \frac{P(D|h)P(h)}{P(D)}
\end{equation*}

Wir suchen also eine Hypothese $h^*$ die den Wert $P(h|D)$ maximiert:
\begin{equation*}
    h^* = \arg\max_{h} P(h|D)
\end{equation*}

Die Grundidee des naiven Bayes-Klassifikators ist die Annahme, dass die einzelnen Merkmale unabhängig voneinander sind.

Sei $D = (x^{(1)}, y^{(1)}), \dots, (x^{(m)}, y^{(m)})$ ein Datensatz, $c$ eine Klasse im Merkmalsraum $Z$  und $x=(x_1, \dots,x_n)\in Z_1 \times \dots \times Z_n$ ein neuer Datenpunkt. Dann ist der Naive-Bayes-Klassifikator
\begin{equation*}
    clf_D^{NaiveBayes}(x) = \arg \max_{c\in Z} P(c|D)P(x_1|c,D)P(x_2|c,D) \dots P(x_n|c,D)
\end{equation*}
mit
\begin{equation*}
    P(c|D) = \frac{|\{(z,c)\in D\}|}{|D|}
\end{equation*}
\begin{equation*}
    P(x_i|c,D) = \frac{|\{(z^\prime,c)\in D| z^\prime = (z_1, \dots, z_n), z_i=x_i \}|}{|\{(z,c)\in D\}|} \quad \text{für} \: i = 1, \dots, n
\end{equation*}

Trotz der naiven Unabhängigkeitsannahme führt dieses Verfahren in der Praxis für eine Vielzahl von Anwendungsfällen zu guten Ergebnissen \cite{hand2001idiot}.
Der naive Bayes-Klassifikator wird daher neben der Sentimentanalyse häufig auch für andere textbasierte Klassifikationsprobleme eingesetzt.
Eine typische Anwendung ist z.B. die Spam-Filterung \cite{sahami1998bayesian}.

\subsection{Deep Learning Ansätze}\label{subsec:deep-learning-ansaetze}

Nach Wankhade et al. \cite{wankhade2022survey} haben in den letzten Jahren \textit{Ţransformer} basierte Modelle \gls{lstm} und \gls{cnn} Modelle in der Stimmungsanalyse abgelöst.
Aus diesem Grund haben wir uns für die Verwendung von \textit{Transformer} basierten \gls{llm} entschieden.

Es wurden \gls{bert}-basierte und \textit{DeepSeek}-basierte Modelle als Vertreter der \textit{Transformer}-Modelle ausgewählt.\\
\gls{bert}-basierte Modelle wurden ausgewählt, da diese in der Literatur als besonders erfolgreich in der Stimmungsanalyse beschrieben werden \cite{devlin2018bert}.
Die \gls{bert}-basierten Modelle und Ansätze sind in Unterabschnitt \ref{subsec:bert} beschrieben.\\
Die \textit{DeepSeek}-R1-basierten Modellen wurden im Rahmen des eigenes Ansatzes aufgrund ihrer Aktualität und der allgemein guten Ergebnisse für unsere Problemstellung evaluiert.
Die Modelle und Ansätze sind in Unterabschnitt \ref{subsec:deepseek} beschrieben.

\textit{Transformer} basieren auf der von Vaswani et al. beschriebenen Architektur \textit{Multi-layer Bidirectional Transformer Encoder} \cite{vaswani2017attention}.

\paragraph{\textit{Transformer}-Architektur}
Die Architektur eines \textit{Multi-layer Bidirectional Transformer Encoder} besteht aus einen \textit{Encoder}, der eine Eingabe von Symbolen $(x_1,...,x_n)$ auf eine kontinuierliche Repräsentation $(z_1,...,z_n)$ abbildet und einem \textit{Decoder}, der aus dieser kontinuierlichen Repräsentation eine Ausgabesequenz aus Symbolen $(y_1,...,y_n)$ generiert.
In jedem Schritt verwendet der \textit{Decoder} seine Ausgabe aus dem vorherigen Schritt.
\textit{Encoder} besteht aus $N$ gleichen Schichten, die jeweils aus einem \textit{Multi-head Self-Attention}-Mechanismus und einem voll-vernetzten Neuronalen Netz bestehen.
Der \textit{Decoder} besteht aus $N$ gleichen Schichten, bestehend aus einer Schicht \textit{Multi-head Self-Attention} und einem voll-vernetzten Neuronalen Netz.
Zusätzlich kommt eine weitere Schicht \textit{Multi-head Self-Attention}, welche die Ausgabe des \textit{Encoders} verarbeitet.

\paragraph{Training von \textit{Transformern}}

Das Training von Transformer-Modellen besteht aus zwei Schritten, dem \textit{Pre-Training} und dem \textit{Finetuning} \cite{Radford2018ImprovingLU}.
Beim \textit{Pre-Training} wird das Modell un-überwacht mit großen Datenmengen trainiert.
Im zweiten Schritt, dem \textit{Finetuning} wird das Modell mit überwachten Lernverfahren oder \textit{Reinforcement Learning} \cite{devlin2018bert, deepseekai2025deepseekr1incentivizingreasoningcapability} auf den jeweiligen Anwendungsfall angepasst.


\subsubsection{\textit{BERT}-basierte Ansätze}\label{subsec:bert}

\gls{bert} ist eine \textit{Transformer}-basierte \gls{llm} Variante, die von Devlin et al. entwickelt wurde \cite{devlin2018bert}.
Devlin et al. haben gezeigt, dass die \gls{bert}-Modelle sehr gute Ergebnisse in der Stimmungsanalyse erzielen und durch \textit{Finetuning} mit vergleichsweise wenig Daten und Trainingszeit auf spezifische Anwendungen angepasst werden kann.
Aufbauend auf \gls{bert} wurde von Liu et al. \cite{liu2019roberta} das Modell \gls{roberta} entwickelt, das durch Optimierungen im \textit{Pre-Training} und \textit{Finetuning} für viele Aufgaben bessere Ergebnisse erzielt.

Von Barbieri et al. \cite{barbieri2020tweeteval} wurde gezeigt, dass für Twitter-spezifische Klassifikationsprobleme \gls{roberta}-basierte Modelle durch \textit{Finetuning} auf Twitter-Daten bessere Ergebnisse erzielen.
Aus diesem Grund haben wir uns für die Verwendung des von Barbieri et al. entwickelten Modells entschieden, welches bereits mittels \textit{Finetuning} an Twitter-Daten angepasst wurde.\\
Für dieses Modell wurden zwei Varianten evaluiert: Eine Verwendung ohne Anpassung auf dem Testdatensatz und eine Anpassung mittels \textit{Finetuning} auf dem Trainingsdatensatz mit anschließender Evaluation auf dem Testdatensatz.

Zum Vergleich haben wir das von \gls{bert} destillierte Modell \textit{DistillBert} \cite{sanh2019distilbert} verwendet.
Bei der Modelldestillation wird ein kleines Modell darauf trainiert, das Verhalten eines großen Modells zu replizieren \cite{sanh2019distilbert}.
Auf diesem Modell wurde ebenfalls ein \textit{Finetuning} auf dem Trainingsdatensatz und eine Evaluation auf dem Testdatensatz durchgeführt.


\subsubsection{\textit{DeepSeek}-basierte Ansätze}\label{subsec:deepseek}

Als eigenen Ansatz haben wir verschiedene vom \textit{DeepSeek}-R1-Modell destillierte Modelle \cite{deepseekai2025deepseekr1incentivizingreasoningcapability} verwendet.
Diese wurden in einem ersten Ansatz versucht, mit \textit{Finetuning} auf unseren Datensatz anzupassen.
Aufgrund der hohen Anforderungen an die Hardware und der langen Trainingszeiten haben wir uns für einen zweiten Ansatz entschieden\footnote{Das \textit{Finetuning} konnten wir lediglich für das kleinste destillierte Modell ausführen.}.\\
In einem zweiten Ansatz haben wir die \textit{DeepSeek}-R1-Modelle mittels \textit{Ollama}\footnote{https://ollama.com/} ausgeführt und mittels Prompt-Anfragen eine Klassifizierung der Tweets durch die Modelle durchgeführt.
Die Prompt-Anfragen wurden dabei mit und ohne Verwendung des Query-Ausdrucks des Testdatensatzes ausgeführt.


\section{Experimente}
% Wie ist der Experimentaufbau, welche Evaluationsmetriken betrachten Sie?
% Beschreibung und Interpretation der Ergebnisse

\subsection{Experimentaufbau}

Für die Experimente wurden unterschiedliche Schritte in Abhängigkeit der Verfahren durchgeführt.

Für die klassischen Verfahren wurde neben der Genauigkeit der Klassifikation für unterschiedliche Modelle auch analysiert, inwiefern Vorverarbeitungsschritte und Token\-isier\-ungs\-verfahren die Genauigkeit der Klassifikation beeinflussen.
Die Schritte für die Experimente sind in Unterabschnitt~\ref{sec:klassische-ansaetze} beschrieben.

Für die \gls{dl} Ansätze, bei denen \gls{bert}-basierte Modelle verwendet wurden, wurden die Standard-Tokenizer der \textit{Hugging Face} Modelle verwendet, so dass keine weiteren Vorverarbeitungsschritte oder Vektorisierungsverfahren durchgeführt wurden.
Die Schritte für die Experimente sind in Unterabschnitt~\ref{sec:deep-learning-ansaetze} beschrieben.

\subsubsection{Klassische Ansätze}\label{sec:klassische-ansaetze}

Für die klassischen Ansätze wurden Sensitivitätsanalysen durchgeführt, um die Auswirkung unterschiedlicher Vorverarbeitungsschritte und Vektorisierungsverfahren auf die Genauigkeit zu analysieren.


\paragraph{Datenvorverarbeitung}
Im Rahmen der Datenvorverarbeitung wurden die verwendeten Stoppwörter und Normalisierungsverfahren variiert.

Nach \cite[S.27]{manning2009introduction} werden unter dem Begriff Stoppwörter\footnote{Beispiele sind \textit{und} sowie \textit{oder}.} Wörter verstanden, die einen geringen Informationsgehalt haben und deshalb aus Texten entfernt werden.
Es wurden die folgenden Ausprägungen von Stoppwortlisten verwendet:
% TODO: Abschließend entscheiden - Listen ggf. durch platzsparendere Tabelle ersetzen
\begin{itemize}
    \item keine Stoppwortliste
    \item Standard-\gls{nltk} Stoppwortliste
    \item Eigene Stoppwortliste
\end{itemize}

Normalisierungsverfahren dienen dazu, die Worte oder Token in Texten zu vereinheitlichen \cite[S.28]{manning2009introduction}.
Es wurden die folgenden Normalisierungsverfahren angewandt:
\begin{itemize}
    \item Keine Normalisierung
    \item Lemmatisierung mit \textit{WordNet} Lemmatizer
    \item Stemming mit \textit{Porter} Stemmer
\end{itemize}

Das Training der Modelle wurde auf Basis der vorverarbeiteten Daten durchgeführt.
Zur Vorverarbeitung der Daten wurde Algorithmus~\ref{alg:data-preparation} durchgeführt.

\paragraph{Training und Evaluation}
Auf Basis der vorverarbeiteten Daten wurden die Modelle trainiert und evaluiert.

Die Texte der vorverarbeiteten Tweets werden mittels Vektorisierungsverfahren in numerische Repräsentationen bzw. Vektoren transformiert.
Aus den Vektoren der einzelnen Tweets setzt sich das Vokabular zusammen, das die Worte der Tweets enthält.
Es wurden zwei Vektorisierungsverfahren verwendet:
\begin{itemize}
    \item \gls{tfidf}-Vektorisierung (mittels \textit{TfidfVectorizer} in der \textit{Python}-Bibliothek \textit{scikit-learn})
    \item Hash-basierte Vektorisierung (mittels \textit{HashingVectorizer} in der \textit{Python}-Bibliothek \textit{scikit-learn})
\end{itemize}
Für die verwendeten Vektorisierungsverfahren wurden unterschiedliche Konfigurationen von n-Grammen verwendet.
n-Gramme bezeichnen nach \cite[S.33]{jm3}, die Sequenz von $n$ aufeinanderfolgenden Wörtern.
Für die Vektorisierungsverfahren wurden Instanzen mit Kombinationen von 1-Grammen, 2-Grammen und 3-Grammen\footnote{Diese n-Gramme werden auch als Uni-, Bi- und Trigramme bezeichnet.} verwendet.

Während des Trainings werden die Modelle auf den Trainingsdaten trainiert und auf den Validierungsdaten evaluiert.
Die Genauigkeit der Modelle wird abschließend auf den Testdaten evaluiert.
Algorithmus~\ref{alg:model-training} beschreibt die Schritte für das Training und die Evaluation der Modelle.

\subsubsection{Deep Learning Ansätze} \label{sec:deep-learning-ansaetze}
Die BERT-Modelle \textit{twitter-roberta-base-sentiment} und \textit{distilbert-base-uncased} wurden über die \textit{Hugging Face}-Python-Bibliothek trainiert und ausgeführt.
Für das \textit{Finetuning} wurde die Standardkonfiguration der Bibliothek mit den Parametern Datensatzgröße und Lernrate in allen Kombinationen mit den Werten aus Tabelle \ref{tab:dl-params} verwendet.
Die Werte orientieren sich an dem Vorgehen von Barbieri et al. \cite{barbieri2020tweeteval}.
\begin{table}
    \begin{tabular}{ll}
        \toprule
        Parameter       & Werte                                                   \\
        \midrule
        Datensatzgröße  & 2500, 5000, 7500, 10000, 15000, 20000                   \\
        Lernrate        & $1e^{-4}$, $5e^{-5}$ , $1e^{-5}$, $5e^{-6}$, $1e^{-6}$  \\
        \bottomrule
    \end{tabular}
    \caption{Parameter für das \textit{Finetuning} der \gls{bert}-Modelle}
    \label{tab:dl-params}
\end{table}


Für den eigenen Ansatz haben wir zuerst versucht, die von Deepseek-R1 destillierten Modelle durch \textit{Finetuning} auf den Datensatz zu trainieren. Dafür wurde dem Modell eine zusätzliche Schicht in Form eines KNN mit einer Eingabeschicht mit der Ausgabe des Modells als Eingabe und zwei Neuronen für die Sentimentklassifizierung als Ausgabe.
Das war für für Deepseek-R1:1.5B noch möglich. Ab dem Modell Deepseek-R1:7B wurden die Hardware-Anforderungen zu groß. (Vgl. mit SGD Optimierung: 7Mrd. Parameter * 2 Byte je Gewicht * 2 Byte je Grandient = 26GB)
Als weiteren Ansatz verwendeten wir die Deepseek-R1-Modelle nur in der Ausführung und ließen das Sentiment per Prompt klassifizieren. Die modelle wurden mit Ollama ausgeführt und per Pythonskript angefragt. Dabei verwendet wir den im Testdatensatz enthaltenen Query-Term um zusätzlich einen aspekt-basierten Ansatz zu testen.
Die Modelle wurden jeweils mit und ohne dem Query-Term angefragt. Eine Anfrage mit dem Query-Term hat folgende Struktur:

\textbf{Prompt:} \greyhl{Tweet sentiment? Sentiment Topic: \{Query-Term\} \\
  Answer with positive or negative. Provide reasoning in JSON.\\
    Tweet: \glqq\{tweet\}\grqq}

\subsection{Modell-Parameter und Evaluationsmetriken}

Für die ausgewählten Modelle wurden die Standard-Parameterwerte von \textit{scikit-learn} verwendet.

Nach~\cite{wankhade2022survey} werden für die Evaluierung von Klassifikationsmodellen vor allem das \textit{Genauig\-keits\-maß}, die \textit{Präzision} oder das \textit{F1-Maß} verwendet.
Weil die Klassenverteilung für die Trainingsdaten ausgeglichen ist, wurde aufgrund der einfachen Interpretierbarkeit das \textit{Genauigkeitsmaß} als Evaluationsmetrik verwendet.

\subsection{Ergebnisse}

\subsubsection{Klassische Ansätze}

\begin{table}
    \begin{tabular}{lllllllrr}
        \toprule
        Modell & Normalisierung & Stoppwortliste   & Anz. Merkmale & n-Gramme & Genauigkeit \\
        \midrule
        LR     & Porter         & eig. Liste       & 250.000       & (1,2)    & 0.861       \\
        LR     & WordNet        & -                & maximal       & (1,3)    & 0.858       \\
        SVM    & Porter         & -                & maximal       & (1,3)    & 0.858       \\
        SVM    & WordNet        & -                & maximal       & (1,3)    & 0.858       \\
        SVM    & Porter         & eig. Liste       & 50.000        & (1,2)    & 0.858       \\
        LR     & Porter         & eig. Liste       & 250.000       & (1,3)    & 0.855       \\
        LR     & -              & \gls{nltk} Liste & maximal       & (1,3)    & 0.855       \\
        SVM    & Porter         & -                & maximal       & (1,2)    & 0.855       \\
        LR     & Porter         & eig. Liste       & 50.000        & (1,2)    & 0.852       \\
        SVM    & -              & -                & maximal       & (1,3)    & 0.852       \\
        NB     & -              & -                & maximal       & (1,2)    & 0.852       \\
        \bottomrule
    \end{tabular}
    \caption{
        Top 10 Modelle nach Testgenauigkeit.
        Die Modelle \textit{LR}, \textit{SVM} und \textit{NB} bezeichnen die Modelle \textit{Logistische Regression}, \textit{Support Vector Machine} und \textit{Naiver Bayes Klassifikator}.
        Die Bezeichner (1, k) in der Spalte \textit{n-Gramme} geben an, dass n-Gramme mit $N\in\lbrace1,\cdots,k\rbrace$ verwendet wurden.
    }
    \label{tab:top-10-models}
\end{table}

In Tabelle~\ref{tab:top-10-models} sind die Top 10 Modelle nach Testgenauigkeit aufgeführt.
Die maximal erzielte Genauigkeit beträgt $0,861$.
Diese Genauigkeit wurde für ein Modell erzielt, bei dem die Normalisierung der Worte mit dem \textit{Porter} Stemmer durchgeführt wurde, die Stoppwörter gemäß einer eigens erstellten Stoppwortliste entfernt wurden, die Anzahl der Merkmale auf $250.000$ beschränkt wurde und Unigramme und Bigramme berücksichtigt wurden.

\paragraph{Sensitivität Modell}
In Tabelle~\ref{tab:stats-per-model} sind die Statistiken der Test Genauigkeit für die Modelle \textit{LinearSVC}, \textit{LogisticRegression} und \textit{NaiveBayes} über alle Parameter-Kombinationen aufgeführt.

Die Logistische Regressions-Modelle erzielen im Mittel die höchsten Genauigkeiten über alle Parameter-Kombinationen.
Weiterhin ist die Standardabweichung der Genauigkeiten für die Logistische Regression mit $0,022$ am niedrigsten.

Modelle auf Basis der \gls{svm} erzielen im Mittel die zweithöchsten Genauigkeiten, wobei Mittelwert und Median $~1\%$ niedriger liegen.

Die Naive Bayes Modelle erzielen im Mittel die niedrigsten Genauigkeiten, wobei die Standardabweichung mit $0,049$ am höchsten ist.
\begin{table}
    \center

    \begin{tabular}{lccccc}
        \toprule
        & \multicolumn{5}{c}{Test Genauigkeit} \\
        Modell             & Mittelwert & Median & Std.-Abweichung & Minimum & Maximum \\
        \midrule
        LogisticRegression & 0.818      & 0.820  & 0.022           & 0.727   & 0.861   \\
        LinearSVC          & 0.809      & 0.813  & 0.024           & 0.730   & 0.858   \\
        NaiveBayes         & 0.778      & 0.784  & 0.049           & 0.685   & 0.852   \\
        \bottomrule
    \end{tabular}
    \caption{Statistiken der Test Genauigkeit für die Modelle \textit{LinearSVC}, \textit{LogisticRegression} und \textit{NaiveBayes} über alle Parameter-Kombinationen.}
    \label{tab:stats-per-model}
\end{table}

\paragraph{Sensitivität Umgang mit Stoppwörtern}

Über alle Parameter-Kombinationen ist die Genauigkeit im Mittel für die Datensätze mit Entfernung der Stoppwörter höher als für die Modelle ohne Entfernung der Stoppwörter.
Weiterhin ist die Genauigkeit mit den Datensätzes, bei denen die Stoppwörter entfernt wurden, die in der eigens definierten Stoppwortliste enthalten sind, höher als bei Verwendung der \gls{nltk} Stoppwortliste.

\paragraph{Sensitivität Normalisierungsverfahren}

Für die unterschiedlichen Normalisierungsverfahren ergeben sich keine signifikanten Unterschiede in der Genauigkeit der Modelle.

\paragraph{Sensitivität Vektorisierungsverfahren und n-Gramme}

Für alle Modelle ist die Genauigkeit höher, wenn der \gls{tfidf}-Vektorizer, anstelle des Hash-basierten verwendet wird

Für alle Modelle steigt die Genauigkeit mit der Anzahl an berücksichtigten Merkmalen.

Für alle Modelle ist die Genauigkeit höher, wenn zusätzlich Bi- und Trigramme berücksichtigt werden.
Die Berücksichtigung von Bi- und Trigrammen führt im Mittel zu den höchsten Genauigkeiten.

\subsubsection{Deep Learning Ansätze}


\section{Ausblick}\label{sec:ausblick}

Es gibt mehrere Ansätze, um die Vorgehensweise und die Ergebnisse der Stimmungsanalyse zu optimieren.

\subsection{\textit{Noisy Label}}
Wie in Abschnitt \ref{subsec:trainingsdatensatz} beschrieben, führt die Art, wie der Trainingsdatensatz \textit{Sentiment140} erstellt wurde, zu \textit{Noisy Label}.
Dies kann bei der Verwendung einfacherer Modelle zu schlechteren Klassifikationsergebnissen führen sowie die Anzahl der benötigten Trainingsbeispiele oder die Komplexität der Modelle erhöhen \cite{NoisyLabel2014}, wie man auch an den moderaten Ergebnissen der verwendeten klassischen Verfahren ($\approx{85}\%$) und den Resultaten von Go et al. ($79-83\%$) sieht (vgl. Tabelle 6 in \cite{go2009twitter}).

Um mit \textit{Noisy Labels} umzugehen, können robustere Modelle oder Methoden der Datenbereinigung verwendet werden, also Datensätze zu entfernen oder sogenannte Semi-überwachte Lernverfahren anzuwenden. Alternativ können auch \textit{Label Noise} tolerante Lernverfahren angewandt werden (vgl. Abschnitt 3 in \cite{NoisyLabel2014}).

Eine grundsätzlich andere Herangehensweise ist die Verwendung eines Datensatzes mit höherer Datenqualität, um \textit{Noisy Labels} zu vermeiden.

\subsection{\textit{Aspect Based Sentiment Analysis}}

Die erlangten Ergebnisse der unterschiedichen Modelle können durch die Verwendung zusätzlicher Informationen verbessert werden, wie die Ergebnisse der Experimente mit den \textit{DeepSeek}-basierten Modellen zeigen.

Bei der \textit{Aspect Based Sentiment Analysis} werden explizite oder implizite Begriffe (\textit{aspects}) aus dem Tweet extrahiert und die Stimmung des Tweets bezüglich dieser Begriffe klassifiziert \cite{Hua_2024}.

Ein Beispiel aus \cite{Hua_2024} zeigt, dass der Satz \glqq \textit{The restaurant was expensive, but the menu was great.}\grqq{} bezüglich dem expliziten Begriff \textit{menu} positiv und bezüglich dem impliziten Begriff \textit{price} negativ klassifiziert wird.

\subsection{Große \textit{Deep Learning} Modelle}

Die Feinabstimmung großer \textit{DeepSeek} Modelle mit dem \textit{Sentiment140} Datensatz stieß auf Rechenkapazitätsgrenzen.
Während das Modell \textit{DeepSeek-R1-1.5B} noch abgestimmt werden konnte, waren die Speicheranforderung während des Trainings für die größeren Modelle mit 8, 32 oder 70 Mrd. Parametern zu hoch für die zur Verfügung stehenden Rechenressourcen.

Mehr Rechenkapazitäten oder Methoden wie  die \textit{Low Rank Adaptation} (LoRA) \cite{lora2021}, bei der nur eine kleine Anzahl neuer Parameter zur Feinabstimmung genutzt wird, können hier Abhilfe schaffen.



\section{Zusammenfassung und Fazit}

In diesem Projekt wurden verschiedene Methoden des maschinellen Lernens zur Stimmungsanalyse auf dem Sentiment140-Twitter-Datensatz evaluiert.
Drei klassische Algorithmen – logistische Regression, \gls{svm} und Naive Bayes-Klassifikator – erreichten maximale Genauigkeiten von bis zu 86\%.
Zwei Transformer-Modelle (\emph{distilbert-base-uncased} und \emph{twitter-roberta-base-sentiment}) wurden durch Fine-Tuning angepasst und erzielten Genauigkeiten von 85\% bzw. 92\%.
Zusätzlich wurde versucht, die durch Modelldestillation erzeugten \emph{Deepseek-R1}-Modelle mittels Fine-Tuning anzupassen. Aufgrund der benötigten Hardware-Ressourcen war dies lediglich für die kleinste Variante (\emph{Deepseek-R1-Distill-Qwen-1.5B}) möglich.
Das größte verwendete \emph{Deepseek-R1} basierte Modell erreichte die höchste Genauigkeit mit 93\% im Zero-Shot-Ansatz.
Wenn zusätzlich ein aspektbasierter Ansatz verwendet wird, erreichen die \emph{Deepseek-R1} basierten Modell-Genauigkeiten bis zu 98\%.

Die Ergebnisse zeigen, dass domänenspezifische Transformer-Modelle bei der Sentiment-Analyse höhere Genauigkeiten erzielen als die analysierten klassischen maschinellen Lernverfahren.
Obwohl die LLM die höchsten Genauigkeiten erreichten, kann in der Praxis der Einsatz vortrainierter Transformer-Modelle (z. B. \emph{twitter-roberta}) für Twitter-Stimmungsanalysen vorteilhafter sein – insbesondere bei begrenzten Hardware-Ressourcen oder dem Einsatz in Anwendungen mit strengen Latenzanforderungen.


% References
\newpage
\addreferences

\makestatement{5}

% ==== Appendix ====
\appendix

\section{Algorithmen}

\subsection{Datenaufbereitung}\label{subsec:appendix-data-preparation}

\textit{Hinweis}: Die Datensätze enthalten weitere Daten je Tweet, wie beispielsweise die Sentiment Klasse.
Der Umgang mit diesen Daten wurde hier aus Gründen der Übersichtlichkeit nicht aufgeführt.

Für die unterschiedlichen Parameter-Ausprägungen für die Normalisierungsverfahren und Stoppwortlisten wurden separate Datensätze erstellt.

\subsubsection{Datenaufbereitung - Algorithmus}
Die Datensätze wurden gemäß des folgenden Algorithmus erzeugt.
\begin{algorithm}
    \caption{Datenaufbereitung}
    \begin{algorithmic}[1]
        \Procedure{DatasetPreparation}{$dataset, normalizerFunction, stoppwords$}
            \Function{SanitizeTweet}{$text$}
                \State $text$ $\gets$ Entferne URLs, Nutzer, Hashtags und Sonderzeichen aus $text$
                \State \Return $text$
            \EndFunction

            \Function{NormalizeTweet}{$text, normalizerFunction, stoppwords$}
                \State $tokens$ $\gets$ Zerlege $text$ mit TweetTokenizer in Token
                \For{$i \gets 1$ \textbf{to} $|tokens|$}
                    \State $tokens$[$i$] $\gets$ $normalizerFunction$($tokens$[$i$])
                \EndFor
                \State $tokens$ $\gets$ Entferne Stoppwörter aus $tokens$ gemäß Stoppwortliste $stoppwords$
                \State $text$ $\gets$ Füge Elemente aus $tokens$ zu einem Text zusammen
                \State \Return $text$
            \EndFunction

            \For{$i \gets 1$ \textbf{to} $|dataset|$}
                \State $dataset[i]$ $\gets$ \Call{SanitizeTweet}{$dataset[i]$}
                \State $dataset[i]$ $\gets$ \Call{NormalizeTweet}{$dataset[i], normalizerFunction, stoppwords$}
            \EndFor
            \State save $dataset$
        \EndProcedure
    \end{algorithmic}
    \label{alg:data-preparation}
\end{algorithm}

\subsubsection{Datenaufbereitung - Beispiel}

Im folgenden ist ein Beispiel für die Datenaufbereitung eines Tweets dargestellt.
\begin{itemize}
    \item \textbf{Original-Tweet}: \textit{\glqq @user I love this movie! http://example.com\grqq}
    \item \textbf{Bereinigung}: \textit{\glqq I love this movie\grqq}
    \item \textbf{Tokenisierung}: \textit{\glqq I\grqq, \glqq love\grqq, \glqq this\grqq, \glqq movie\grqq}
    \item \textbf{Normalisierung}:
    \begin{itemize}
        \item \textit{Lemmatisierung}: \textit{\glqq I\grqq, \glqq love\grqq, \glqq this\grqq, \glqq movie\grqq}
        \item \textit{Stemming}: \textit{\glqq I\grqq, \glqq lov\grqq, \glqq thi\grqq, \glqq movi\grqq}
    \end{itemize}
    \item \textbf{Stoppwörter Behandlung}: Ohne Stoppwörter \glqq{}I\grqq{} und \glqq{}this\grqq{}: \textit{\glqq love\grqq, \glqq movie\grqq}
    \item \textbf{Aufbereiteter Text}: \textit{\glqq love movie\grqq}
\end{itemize}

\subsection{Training und Evaluation der Modelle}

Die Schritte für das Training und die Evaluation der Modelle sind, aufbauend auf dem durch Algorithmus \ref{alg:data-preparation} erzeugten Datensätzen wie folgt:
\begin{algorithm}
    \caption{Training und Evaluation der Modelle}
    \begin{algorithmic}[1]
        \Procedure{ModelTraining}{$dataset, vectorizer, model$}
            \State $X, y \gets$ Extrahiere Texte und Labels aus $dataset$
            \State $X \gets$ Transformiere Texte in $X$ in Vektoren mit Vektorizer $tokenizer$
            \State $X_{train}, X_{test}, y_{train}, y_{test} \gets$ Teile $X$ und $y$ in Trainings- und Testdaten
            \State $model \gets$ Trainiere $model$ auf $X_{train}$ und $y_{train}$
            \State $accuracy \gets$ Evaluiere $model$ auf $X_{test}$ und $y_{test}$
            \State \Return $accuracy$
        \EndProcedure
    \end{algorithmic}
    \label{alg:model-training}
\end{algorithm}




\end{document}
