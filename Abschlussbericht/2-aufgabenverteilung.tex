\section{Aufgabenverteilung}\label{sec:aufgabenverteilung}
% Ein Abschnitt pro Teammitglied
% Kurze Übersicht, was die Person innerhalb des Praktikums beigetragen hat, insbesondere
% entwickelter Code, Beitrag zum Abschlussbericht, organisatorischer Beitrag, Beitrag zur
% Abschlusspräsentation, etc.

\begin{multicols}{2}
[
Die Themen Literaturrecherche und Ideenfindung, Überprüfung der Datenqualität des gegebenen Datensatzes, Suche nach einem neuen Datensatz und die explorative Datenanalyse des \textit{Sentiment140} Datensatzes wurden von allen Projektmitgliedern bearbeitet.
Darüber hinaus wurden von allen Projektmitgliedern ein klassischen Verfahren angewandt und unterschiedliche Kombinationen von Vorverarbeitungsschritten in Bezug auf die Genauigkeit getestet.
Außerdem wurde die Zwischen- und Abschlusspräsentation nach der initialen Erstellung gemeinsam überarbeitet.
Der Abschnitte des Abschlussberichts wurden anfänglich von einzelnen Mitgliedern geschrieben und anschließend gemeinsam überarbeitet.
Im folgenden sind die Aufgaben aufgezählt, die nicht von allen Mitgliedern durchgeführt wurden.
]

\subsection{Anne Huber}
\begin{itemize}
    \item Kommunikation mit den Projektbetreuern
    \item Protokollerstellung der Treffen
    \item Halten der Zwischenpräsentation
    \item Klassische Verfahren: \gls{knn}
    \item Abschlussbericht: \ref{sec:aufgabenverteilung}, \ref{sec:teaminterneorganisation}, \ref{sec:ausblick}
\end{itemize}


\subsection{Andreas Franke}
\begin{itemize}
    \item Klassische Verfahren: SVM
    \item Implementierung des Frameworks für die klassischen Verfahren
    \item Implementierung \textit{Finetuning}-Logik für die BERT-basierten Ansätze
    \item Implementierung der Prompting-Logik für die \textit{DeepSeek} Modelle
    \item Durchführung Komplett-Abläufe klassische Verfahren
    \item Durchführung Abläufe für die BERT-basierten Ansätze
    \item Durchführung der \textit{Zero-Shot}-Ansätze (\textit{DeepSeek} R1 1.5B, 8B, 32B)
    \item Abschlussbericht: \ref{sec:datensätzeundproblemstellung}, \ref{subsec:klassische-ansaetze}, \ref{subsubsec:experimente-klassische-ansaetze}, \ref{subsec:modell-parameter-und-evaluationsmetriken}, \ref{subsubsec:ergebnisse-klassische-ansaetze}
\end{itemize}

\subsection{Felix Lindner}
\begin{itemize}
    \item Klassische Verfahren: Entscheidungsbäume
    \item Erweiterung des Frameworks für die klassischen Verfahren
    \item Implementierung \textit{Finetuning}-Logik für die BERT-basierten Ansätze
    \item Durchführung Abläufe für die BERT-basierten Ansätze
    \item Durchführung des \textit{Zero-Shot}-Ansatzes (\textit{DeepSeek} R1 70B)
    \item Halten der Abschlusspräsentation
    \item Abschlussbericht: \ref{subsec:deep-learning-ansaetze}, \ref{subsubsec:experimente-deep-learning-ansaetze}, \ref{subsubsec:ergebnisse-deep-learning-ansaetze}
\end{itemize}

\subsection{Burak Özkan}
\begin{itemize}
    \item Klassische Verfahren: Naiver Bayes Klassifikator
    \item Erstellung der Abschlusspräsentation
    \item Abschlussbericht: \ref{sec:zusammenfassung-und-fazit}
\end{itemize}

\subsection{Milomir Soknic}
\begin{itemize}
    \item Klassische Verfahren: Logistische Regression
    \item Erstellung der Zwischenpräsentation
    \item Erstellung der Abschlusspräsentation
    \item Abschlussbericht: \ref{sec:einleitung}
\end{itemize}

\end{multicols}
