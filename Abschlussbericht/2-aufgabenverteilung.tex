\section{Aufgabenverteilung}\label{sec:aufgabenverteilung}
% Ein Abschnitt pro Teammitglied
% Kurze Übersicht, was die Person innerhalb des Praktikums beigetragen hat, insbesondere
% entwickelter Code, Beitrag zum Abschlussbericht, organisatorischer Beitrag, Beitrag zur
% Abschlusspräsentation, etc.

\begin{multicols}{2}
[
Die Themen Literaturrecherche und Ideenfindung, Überprüfung der Datenqualität des gegebenen Datensatzes, Suche nach einem neuen Datensatz und die explorative Datenanalyse des \textit{Sentiment140} Datensatzes wurden von allen Projektmitgliedern bearbeitet.
Darüber hinaus wurden von allen Projektmitgliedern ein klassischen Verfahren angewandt und unterschiedliche Kombinationen von Vorverarbeitungsschritten in Bezug auf die Genauigkeit getestet.
Außerdem wurde die Zwischen- und Abschlusspräsentation nach der initialen Erstellung gemeinsam überarbeitet.
Der Abschnitte des Abschlussberichts wurden initial von einzelnen Mitgliedern geschrieben und anschließend gemeinsam überarbeitet.\\
Im folgenden sind die Aufgaben aufgezählt, die nicht von allen Mitgliedern durchgeführt wurden.\\
\redhl{systematischere Aufzählung der Aufgaben}\\
\redhl{Verweis auf Abschnitte/Unterabschnitte mit \textit{ref} und \textit{label}?}
]

\subsection{Anne Huber}
\begin{itemize}
    \item Moderator der Gruppe
    \item Kommunikation nach Außen
    \item Schreiben der Protokolle
    \item Halten der Zwischenpräsentation
    \item Klassische Verfahren: \gls{knn}
    \item Abschlussbericht: 2. Aufgabenverteilung, 3. Teaminterne Organisation, 7. Ausblick
\end{itemize}


\subsection{Andreas Franke}
\begin{itemize}
    \item Klassische Verfahren: \gls{svm}
    \item Erstellung eines Frameworks zur Durchführung und zum Vergleich der klassischen Ansätze
    \item Durchführung des \textit{Zero-Shot}-Ansatzes (\textit{DeepSeek} R1 1.5B, 32B)
    \item Abschlussbericht: 4. Datensätze und Problemstellung, 5.1.2 SVM, 6. Experimente
\end{itemize}

\subsection{Felix Lindner}
\begin{itemize}
    \item Klassische Verfahren: Entscheidungsbäume/Entscheidungswälder
    \item Refactoring und Vereinheitlichung des Frameworks der klassischen Verfahren
    \item Anwendung des \textit{Deep Learning} Ansatzes (\textit{DistilBERT}, \textit{RoBERTa})
    \item Durchführung des \textit{Zero-Shot}-Ansatzes (\textit{DeepSeek} R1 70B)
    \item Halten der Abschlusspräsentation
    \item Abschlussbericht: 5.2 Deep Learning
\end{itemize}

\subsection{Burak Özkan}
\begin{itemize}
    \item Klassische Verfahren: Naiver Bayes Klassifikator
    \item Finden des Datensatzes \textit{Sentiment140}
    \item Erstellen der finalen Abschlusspräsentation
    \item Abschlussbericht: 5.1.3 Naiver Bayes Klassifikator, 8. Zusammenfassung und Fazit
\end{itemize}

\subsection{Milomir Soknic}
\begin{itemize}
    \item Klassische Verfahren: Logistische Regression
    \item Erstellen der Zwischenpräsentation
    \item Erstellen der initialen Abschlusspräsentation
    \item Abschlussbericht: 5.1.1 Logistische Regression, 1. Einleitung
\end{itemize}

\end{multicols}
