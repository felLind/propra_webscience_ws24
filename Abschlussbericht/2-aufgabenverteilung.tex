\section{Aufgabenverteilung}
% Ein Abschnitt pro Teammitglied
% Kurze Übersicht, was die Person innerhalb des Praktikums beigetragen hat, insbesondere
% entwickelter Code, Beitrag zum Abschlussbericht, organisatorischer Beitrag, Beitrag zur
% Abschlusspräsentation, etc.

%==================In zwei Spalten============================================

\begin{multicols}{2}
[
%Die Teammitglieder haben an allen virtuellen Treffen teilgenommen. \\
Aufgaben, die von allen Teilnehmern übernommen wurden, sind die Überprüfung der Datenqualität des gegebenen Datensatzes, die Suche nach einem neuen Datensatz und die exporative Datenanalyse des \textit{Sentiment140} Datensatzes. Darüber hinaus haben alle jeweils ein klassisches maschinelles Lernverfahren angewandt und versucht über unterschiedliche Kombinationen von Vorverarbeitungsschritten möglichst hohe Genauigkeiten zu erhalten. 
Insbesondere haben sich alle in den unterschiedlichen Schritten des Projekt an der Literaturrecherche und Ideenfindung beteiligt.
Außerdem wurde die Zwischen- und Abschlusspräsentation nach der initialen Erstellung gemeinsam überarbeitet und eingeübt. Das Schreiben des Abschlussberichts wurde aufgeteilt und die anschließenden überarbeitenden und anpassenden Schritte des Abschlussberichts wurden ebenfalls gemeinsam von allen übernommen. \\
Im folgenden sind die Aufgaben aufgezählt, die nicht von allen Mitgliedern durchgeführt wurden.
]

\subsection{Anne Huber}
\begin{itemize}
    \item Moderator der Gruppe
    \item Kommunikation nach Außen
    \item Schreiben der Protokolle
    \item Halten der Zwischenpräsentation
    \item Klassische Verfahren: kNN
    \item Abschlussbericht: 2. Aufgabenverteilung, 3. Teaminterne Organisation, 7. Ausblick
\end{itemize}


\subsection{Andreas Franke}
\begin{itemize}
    \item Klassische Verfahren: SVM
    \item Erstellung eines Frameworks zur Durchführung und zum Vegleich der klassischen Ansätze
    \item Durchführung des Zero-Shot-Ansatzes (DeepSeek R1 1.5B, 32B)
    \item Abschlussbericht: 4. Datensätze und Problemstellung, 5.1.2 SVM, 6. Experimente
\end{itemize}

\subsection{Felix Lindner}
\begin{itemize}
    \item Klassische Verfahren: Entscheidungsbäume/Entscheidungswälder
    \item Refactoring und Vereinheitlichung des Frameworks der klassischen Verfahren
    \item Anwendung des \textit{Deep Learning} Ansatzes (DistilBERT, RoBERTa)
    \item Durchführung des Zero-Shot-Ansatzes (DeepSeek R1 70B)
    \item Halten der Abschlusspräsentation
    \item Abschlussbericht: 5.2 Deep Learning
\end{itemize}

\subsection{Burak Özkan}
\begin{itemize}
    \item Klassische Verfahren: Naiver Bayes Klassifikator
    \item Finden des Datensatzes \textit{Sentiment140}
    \item Erstellen der finalen Abschlusspräsentation
    \item Abschlussbericht: 5.1.3 Naiver Bayes Klassifikator, 8. Zusammenfassung und Fazit
\end{itemize}

\subsection{Milomir Soknic}
\begin{itemize}
    \item Klassische Verfahren: Logistische Regression
    \item Erstellen der Zwischenpräsentation
    \item Erstellen der initialen Abschlusspräsentation
    \item Abschlussbericht: 5.1.1 Logistische Regression
\end{itemize}

\end{multicols}
