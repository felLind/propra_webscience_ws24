\section{Teaminterne Organisation}
% Wie wurde innerhalb des Teams kommuniziert?
% Welche Programmiersprache? Warum?
% Welche Tools/Techniken wurden verwendet?
% etc.

Der erste virtuelle Kontakt fand über Zoom statt. Hier wurde, mit Zustimmung aller Gruppenmitglieder, beschlossen, eine private Lerngruppe auf Discord zur asynchronen Kommunikation zu verwenden. Gleichzeitig gibt es hier die Möglichkeit einen Sprachkanal zu nutzen, um die regelmäßige Online-Treffen abzuhalten. Die Treffen fanden meist im wöchentlichen Rhythmus statt, je nachdem wie zeitaufwendig die aktuellen Aufgaben der Mitglieder waren. Hier wurden die erreichten Ergebnisse besprochen und darauf aufbauend unter Mitwirkung aller Teilnehmer die nächsten Schritte besprochen und geplant. Diese wurden im Anschluss in einzelne Aufgaben aufgeteilt.

Als Programmiersprache wurde Python gewählt. Alle Teammitglieder hatten unter anderem durch die vorherige Teilnahme am Modul \textit{Einführung in Maschinelles Lernen} hiermit schon Erfahrung. Insbesondere die Bibliothek \textit{scikit-learn} wurde bei der Anwendung der klassischen maschinellen Lernverfahren angewandt.
Zur Verwaltung der verwendeten Python-Pakete wurde \textit{Poetry} und \textit{pre-commit} verwendet, um eine einheitliche Formatierung der Dateien zu erreichen.

Für die Anwendung der \textit{Deep Learning} Ansätze und die eigenen Ansätze wurde die \textit{transformers} Bibliothek von \textit{Hugging Face} verwendet. Diese bietet eine einfache Möglichkeit auf vortrainierte BERT-basierte Modell zu zugreifen und diese zu trainieren, bzw. zu verfeinern.

Zu Beginn, während der Überprüfung der Datensätze, der explorativen Datenanalyse und der Anwendung der klassischen Verfahren, haben alle Gruppenmitglieder parallel gearbeitet und ihren Code/Ergebnisse jeweils auf einem \textit{branch} auf dem \textit{GitHub Repository} zur Verfügung gestellt. Später, bei der Arbeit mit \textit{Transformer}-Modellen und bei der Erstellung des Abschlussberichts wurden \textit{Pull Requests} erstellt, so dass das Feedback der Anderen eingeholt werden konnte. 

Die Abfolge der einzelnen Projektschritte orientierte sich an dem \textit{Data Science Life Cycle} (\cite{Stodden2020} Abbildung 2). Dieser wurde während der Anwendung der klassischen Methoden einmal durchlaufen. Dies beinhaltet die wiederholte Durchführung von Schritten, wie \textit{Obtain Data, Data Exploration} oder die parallele Ausführung von \textit{Data Preparation} und \textit{Model Estimation}.  Im Anschluss, bei der Anwendung der \textit{Deep Learning} Modelle, wurde gleichermaßen mehrfach über gewisse Schritte des \textit{Data Science Life Cycle} iteriert.
