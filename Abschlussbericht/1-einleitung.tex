\section{Einleitung}

Twitter zählt zu den bedeutendsten sozialen Netzwerken, auf dem täglich Millionen von Nutzenden ihre Meinungen zu unterschiedlichen Themen äußern.
Aufgrund der enormen Menge an Tweets bietet die Plattform wertvolle Einblicke in die öffentliche Meinung und gesellschaftliche Diskurse \cite{pak2010twitter}.
Die Stimmungsanalyse, ein Teilgebiet der \gls{cl}, ermöglicht es, Emotionen und Meinungen automatisiert zu erkennen und zu kategorisieren.
Insbesondere im Bereich der Web Science trägt diese Methode dazu bei, Dynamiken in soziotechnischen Systemen besser zu verstehen \cite{berners2006web, liu2012sentiment}.

Im Rahmen dieses Projekts werden verschiedene Verfahren des maschinellen Lernens auf einen Twitter-Datensatz angewendet, um die mittels Tweets ausgedrückte Stimmung der Nutzer zu analysieren.
Als Datengrundlage dient der \textit{Sentiment140}-Datensatz der auf \textit{HuggingFace}\footnote{https://huggingface.co/datasets/stanfordnlp/sentiment140} verfügbar ist.
Der Datensatz entält eine große Menge annotierter Tweets und wurde von Go et al. erstellt \cite{go2009twitter}.\\
Ziel der Analysen ist es, unterschiedliche Ansätze, einschließlich Ansätze aus dem Bereich \gls{dl}, hinsichtlich ihrer Fähigkeit zur Stimmungsanalyse zu evaluieren.

Im Bericht wird zunächst die Aufgabenverteilung im Team sowie die genutzten Tools und Methoden beschrieben.
Anschließend werden die Schritte der Datenaufbereitung, die eingesetzten Modelle und die durchgeführten Experimente erläutert.
Abschließend erfolgt eine Diskussion der Ergebnisse, gefolgt von einem Ausblick auf Optimierungsmöglichkeiten und weiterführende Forschungsansätze.
