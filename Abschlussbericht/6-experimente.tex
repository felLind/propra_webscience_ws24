\section{Experimente}
% Wie ist der Experimentaufbau, welche Evaluationsmetriken betrachten Sie?
% Beschreibung und Interpretation der Ergebnisse

\subsection{Experimentaufbau}

Für die Experimente wurden unterschiedliche Schritte in Abhängigkeit der Verfahren durchgeführt.

Für die klassischen Verfahren wurde neben der Genauigkeit der Klassifikation für unterschiedliche Modelle auch analysiert, inwiefern Vorverarbeitungsschritte und Token\-isier\-ungs\-verfahren die Genauigkeit der Klassifikation beeinflussen.
Die Schritte für die Experimente sind in Unterabschnitt~\ref{sec:klassische-ansaetze} beschrieben.

Für die \gls{dl} Ansätze, bei denen \gls{bert}-basierte Modelle verwendet wurden, wurden die Standard-Tokenizer der \textit{Hugging Face} Modelle verwendet, so dass keine weiteren Vorverarbeitungsschritte oder Vektorisierungsverfahren durchgeführt wurden.
Die Schritte für die Experimente sind in Unterabschnitt~\ref{sec:deep-learning-ansaetze} beschrieben.

\subsubsection{Klassische Ansätze}\label{sec:klassische-ansaetze}

Für die klassischen Ansätze wurden Sensitivitätsanalysen durchgeführt, um die Auswirkung unterschiedlicher Vorverarbeitungsschritte und Vektorisierungsverfahren auf die Genauigkeit zu analysieren.


\paragraph{Datenvorverarbeitung}
Im Rahmen der Datenvorverarbeitung wurden die verwendeten Stoppwörter und Normalisierungsverfahren variiert.

Nach \cite[S.27]{manning2009introduction} werden unter dem Begriff Stoppwörter\footnote{Beispiele sind \textit{und} sowie \textit{oder}.} Wörter verstanden, die einen geringen Informationsgehalt haben und deshalb aus Texten entfernt werden.
Es wurden die folgenden Ausprägungen von Stoppwortlisten verwendet:
% TODO: Abschließend entscheiden - Listen ggf. durch platzsparendere Tabelle ersetzen
\begin{itemize}
    \item keine Stoppwortliste
    \item Standard-\gls{nltk} Stoppwortliste
    \item Eigene Stoppwortliste
\end{itemize}

Normalisierungsverfahren dienen dazu, die Worte oder Token in Texten zu vereinheitlichen \cite[S.28]{manning2009introduction}.
Es wurden die folgenden Normalisierungsverfahren angewandt:
\begin{itemize}
    \item Keine Normalisierung
    \item Lemmatisierung mit \textit{WordNet} Lemmatizer
    \item Stemming mit \textit{Porter} Stemmer
\end{itemize}

Das Training der Modelle wurde auf Basis der vorverarbeiteten Daten durchgeführt.
Zur Vorverarbeitung der Daten wurde Algorithmus~\ref{alg:data-preparation} durchgeführt.

\paragraph{Training und Evaluation}
Auf Basis der vorverarbeiteten Daten wurden die Modelle trainiert und evaluiert.

Die Texte der vorverarbeiteten Tweets werden mittels Vektorisierungsverfahren in numerische Repräsentationen bzw. Vektoren transformiert.
Aus den Vektoren der einzelnen Tweets setzt sich das Vokabular zusammen, das die Worte der Tweets enthält.
Es wurden zwei Vektorisierungsverfahren verwendet:
\begin{itemize}
    \item \gls{tfidf}-Vektorisierung (mittels \textit{TfidfVectorizer} in der \textit{Python}-Bibliothek \textit{scikit-learn})
    \item Hash-basierte Vektorisierung (mittels \textit{HashingVectorizer} in der \textit{Python}-Bibliothek \textit{scikit-learn})
\end{itemize}
Für die verwendeten Vektorisierungsverfahren wurden unterschiedliche Konfigurationen von n-Grammen verwendet.
n-Gramme bezeichnen nach \cite[S.33]{jm3}, die Sequenz von $n$ aufeinanderfolgenden Wörtern.
Für die Vektorisierungsverfahren wurden Instanzen mit Kombinationen von 1-Grammen, 2-Grammen und 3-Grammen\footnote{Diese n-Gramme werden auch als Uni-, Bi- und Trigramme bezeichnet.} verwendet.

Während des Trainings werden die Modelle auf den Trainingsdaten trainiert und auf den Validierungsdaten evaluiert.
Die Genauigkeit der Modelle wird abschließend auf den Testdaten evaluiert.
Algorithmus~\ref{alg:model-training} beschreibt die Schritte für das Training und die Evaluation der Modelle.

\subsubsection{Deep Learning Ansätze} \label{sec:deep-learning-ansaetze}
Die BERT-Modelle \textit{twitter-roberta-base-sentiment} und \textit{distilbert-base-uncased} wurden über die \textit{Hugging Face}-Python-Bibliothek trainiert und ausgeführt.
Für das \textit{Finetuning} wurde die Standardkonfiguration der Bibliothek mit den Parametern Datensatzgröße und Lernrate in allen Kombinationen mit den Werten aus Tabelle \ref{tab:dl-params} verwendet.
Die Werte orientieren sich an dem Vorgehen von Barbieri et al. \cite{barbieri2020tweeteval}.
\begin{table}
    \begin{tabular}{ll}
        \toprule
        Parameter       & Werte                                                   \\
        \midrule
        Datensatzgröße  & 2500, 5000, 7500, 10000, 15000, 20000                   \\
        Lernrate        & $1e^{-4}$, $5e^{-5}$ , $1e^{-5}$, $5e^{-6}$, $1e^{-6}$  \\
        \bottomrule
    \end{tabular}
    \caption{Parameter für das \textit{Finetuning} der \gls{bert}-Modelle}
    \label{tab:dl-params}
\end{table}


Für den eigenen Ansatz haben wir zuerst versucht, die von Deepseek-R1 destillierten Modelle durch \textit{Finetuning} auf den Datensatz zu trainieren. Dafür wurde dem Modell eine zusätzliche Schicht in Form eines KNN mit einer Eingabeschicht mit der Ausgabe des Modells als Eingabe und zwei Neuronen für die Sentimentklassifizierung als Ausgabe.
Das war für für Deepseek-R1:1.5B noch möglich. Ab dem Modell Deepseek-R1:7B wurden die Hardware-Anforderungen zu groß. (Vgl. mit SGD Optimierung: 7Mrd. Parameter * 2 Byte je Gewicht * 2 Byte je Grandient = 26GB)
Als weiteren Ansatz verwendeten wir die Deepseek-R1-Modelle nur in der Ausführung und ließen das Sentiment per Prompt klassifizieren. Die modelle wurden mit Ollama ausgeführt und per Pythonskript angefragt. Dabei verwendet wir den im Testdatensatz enthaltenen Query-Term um zusätzlich einen aspekt-basierten Ansatz zu testen.
Die Modelle wurden jeweils mit und ohne dem Query-Term angefragt. Eine Anfrage mit dem Query-Term hat folgende Struktur:

\textbf{Prompt:} \greyhl{Tweet sentiment? Sentiment Topic: \{Query-Term\} \\
  Answer with positive or negative. Provide reasoning in JSON.\\
    Tweet: \glqq\{tweet\}\grqq}

\subsection{Modell-Parameter und Evaluationsmetriken}

Für die ausgewählten Modelle wurden die Standard-Parameterwerte von \textit{scikit-learn} verwendet.

Nach~\cite{wankhade2022survey} werden für die Evaluierung von Klassifikationsmodellen vor allem das \textit{Genauig\-keits\-maß}, die \textit{Präzision} oder das \textit{F1-Maß} verwendet.
Weil die Klassenverteilung für die Trainingsdaten ausgeglichen ist, wurde aufgrund der einfachen Interpretierbarkeit das \textit{Genauigkeitsmaß} als Evaluationsmetrik verwendet.

\subsection{Ergebnisse}

\subsubsection{Klassische Ansätze}

\begin{table}
    \begin{tabular}{lllllllrr}
        \toprule
        Modell & Normalisierung & Stoppwortliste   & Anz. Merkmale & n-Gramme & Genauigkeit \\
        \midrule
        LR     & Porter         & eig. Liste       & 250.000       & (1,2)    & 0.861       \\
        LR     & WordNet        & -                & maximal       & (1,3)    & 0.858       \\
        SVM    & Porter         & -                & maximal       & (1,3)    & 0.858       \\
        SVM    & WordNet        & -                & maximal       & (1,3)    & 0.858       \\
        SVM    & Porter         & eig. Liste       & 50.000        & (1,2)    & 0.858       \\
        LR     & Porter         & eig. Liste       & 250.000       & (1,3)    & 0.855       \\
        LR     & -              & \gls{nltk} Liste & maximal       & (1,3)    & 0.855       \\
        SVM    & Porter         & -                & maximal       & (1,2)    & 0.855       \\
        LR     & Porter         & eig. Liste       & 50.000        & (1,2)    & 0.852       \\
        SVM    & -              & -                & maximal       & (1,3)    & 0.852       \\
        NB     & -              & -                & maximal       & (1,2)    & 0.852       \\
        \bottomrule
    \end{tabular}
    \caption{
        Top 10 Modelle nach Testgenauigkeit.
        Die Modelle \textit{LR}, \textit{SVM} und \textit{NB} bezeichnen die Modelle \textit{Logistische Regression}, \textit{Support Vector Machine} und \textit{Naiver Bayes Klassifikator}.
        Die Bezeichner (1, k) in der Spalte \textit{n-Gramme} geben an, dass n-Gramme mit $N\in\lbrace1,\cdots,k\rbrace$ verwendet wurden.
    }
    \label{tab:top-10-models}
\end{table}

In Tabelle~\ref{tab:top-10-models} sind die Top 10 Modelle nach Testgenauigkeit aufgeführt.
Die maximal erzielte Genauigkeit beträgt $0,861$.
Diese Genauigkeit wurde für ein Modell erzielt, bei dem die Normalisierung der Worte mit dem \textit{Porter} Stemmer durchgeführt wurde, die Stoppwörter gemäß einer eigens erstellten Stoppwortliste entfernt wurden, die Anzahl der Merkmale auf $250.000$ beschränkt wurde und Unigramme und Bigramme berücksichtigt wurden.

\paragraph{Sensitivität Modell}
In Tabelle~\ref{tab:stats-per-model} sind die Statistiken der Test Genauigkeit für die Modelle \textit{LinearSVC}, \textit{LogisticRegression} und \textit{NaiveBayes} über alle Parameter-Kombinationen aufgeführt.

Die Logistische Regressions-Modelle erzielen im Mittel die höchsten Genauigkeiten über alle Parameter-Kombinationen.
Weiterhin ist die Standardabweichung der Genauigkeiten für die Logistische Regression mit $0,022$ am niedrigsten.

Modelle auf Basis der \gls{svm} erzielen im Mittel die zweithöchsten Genauigkeiten, wobei Mittelwert und Median $~1\%$ niedriger liegen.

Die Naive Bayes Modelle erzielen im Mittel die niedrigsten Genauigkeiten, wobei die Standardabweichung mit $0,049$ am höchsten ist.
\begin{table}
    \center

    \begin{tabular}{lccccc}
        \toprule
        & \multicolumn{5}{c}{Test Genauigkeit} \\
        Modell             & Mittelwert & Median & Std.-Abweichung & Minimum & Maximum \\
        \midrule
        LogisticRegression & 0.818      & 0.820  & 0.022           & 0.727   & 0.861   \\
        LinearSVC          & 0.809      & 0.813  & 0.024           & 0.730   & 0.858   \\
        NaiveBayes         & 0.778      & 0.784  & 0.049           & 0.685   & 0.852   \\
        \bottomrule
    \end{tabular}
    \caption{Statistiken der Test Genauigkeit für die Modelle \textit{LinearSVC}, \textit{LogisticRegression} und \textit{NaiveBayes} über alle Parameter-Kombinationen.}
    \label{tab:stats-per-model}
\end{table}

\paragraph{Sensitivität Umgang mit Stoppwörtern}

Über alle Parameter-Kombinationen ist die Genauigkeit im Mittel für die Datensätze mit Entfernung der Stoppwörter höher als für die Modelle ohne Entfernung der Stoppwörter.
Weiterhin ist die Genauigkeit mit den Datensätzes, bei denen die Stoppwörter entfernt wurden, die in der eigens definierten Stoppwortliste enthalten sind, höher als bei Verwendung der \gls{nltk} Stoppwortliste.

\paragraph{Sensitivität Normalisierungsverfahren}

Für die unterschiedlichen Normalisierungsverfahren ergeben sich keine signifikanten Unterschiede in der Genauigkeit der Modelle.

\paragraph{Sensitivität Vektorisierungsverfahren und n-Gramme}

Für alle Modelle ist die Genauigkeit höher, wenn der \gls{tfidf}-Vektorizer, anstelle des Hash-basierten verwendet wird

Für alle Modelle steigt die Genauigkeit mit der Anzahl an berücksichtigten Merkmalen.

Für alle Modelle ist die Genauigkeit höher, wenn zusätzlich Bi- und Trigramme berücksichtigt werden.
Die Berücksichtigung von Bi- und Trigrammen führt im Mittel zu den höchsten Genauigkeiten.
$1e^{-4}$, $5e^{-5}$ , $1e^{-5}$, $5e^{-6}$, $1e^{-6}$
\subsubsection{Deep Learning Ansätze}
\begin{table}
    \begin{tabular}{lrrr}
        \toprule
        Modell      & Datensatzgröße & Lernrate  & Genauigkeit \\
        \midrule
        RoBERTA     & -              & -         & 0.830       \\
        RoBERTA     & 7500           & $1e^{-6}$ & 0.922       \\
        RoBERTA     & 2500           & $1e^{-5}$ & 0.919       \\
        RoBERTA     & 2500           & $5e^{-6}$ & 0.919       \\
        RoBERTA     & 5000           & $5e^{-6}$ & 0.919       \\
        DistillBert & -              & -         & 0.490       \\
        DistillBert & 20000          & $5e^{-6}$ & 0.849       \\
        DistillBert & 10000          & $5e^{-5}$ & 0.846       \\
        DistillBert & 10000          & $1e^{-5}$ & 0.846       \\
        DistillBert & 2500           & $1e^{-5}$ & 0.844       \\
        \bottomrule
    \end{tabular}
    \caption{
        Vergleich RoBERTA und DistillBert nach Testgenauigkeit. Jeweils als Baseline und die vier besten Ergebnisse.
    }
    \label{tab:top-models-dl}
\end{table}
Die Tabelle \ref{tab:top-models-dl} zeigt ganz klar, dass die auf den Anwendungsfall angepasst Modelle bessere Ergebnisse liefern, als Basistrainierte und auf den Datensatz feinangepasste Modelle.
