\section{Zusammenfassung und Fazit}\label{sec:zusammenfassung-und-fazit}

In diesem Projekt wurden verschiedene Methoden des maschinellen Lernens zur Stimmungsanalyse auf dem Sentiment140-Twitter-Datensatz evaluiert.
Drei klassische Algorithmen – \gls{lr}, \gls{svm} und \gls{nb}-Klassifikator – erreichten maximale Genauigkeiten von bis zu 85\%.
Zwei Transformer-Modelle (\emph{distilbert-base-uncased} und \emph{twitter-roberta-base-sentiment}) wurden durch \textit{Finetuning} angepasst und erzielten Genauigkeiten von 85\% bzw. 92\%.
Zusätzlich wurde versucht, die durch Modelldestillation erzeugten \emph{DeepSeek-R1}-Modelle mittels \textit{Finetuning} anzupassen.
Aufgrund der benötigten Hardware-Res\-sour\-cen war dies lediglich für die kleinste Variante (\emph{Deepseek-R1-1.5B}) möglich, die Genauigkeiten von bis zu 87\% lieferte.
Das größte verwendete \emph{DeepSeek-R1} basierte Modell erreichte die höchste Genauigkeit mit 93\% im \textit{Zero-Shot}-Ansatz.
Mit einem aspektbasierten Ansatz wurden mit den \emph{DeepSeek-R1} basierten Modellen Genauigkeiten bis zu 98\% erzielt.

Die Ergebnisse zeigen, dass domänenspezifische Transformer-Modelle bei der Stimmungsanalyse höhere Genauigkeiten erzielen als die analysierten klassischen maschinellen Lernverfahren.
Obwohl die LLM mit den meisten Parametern die höchsten Genauigkeiten erreichten, kann in der Praxis der Einsatz vortrainierter kleinerer Transformer-Modelle (z. B. \emph{RoBERTa}) für Twitter-Stimmungsanalysen vorteilhafter sein – insbesondere bei begrenzten Hardware-Ressourcen oder dem Einsatz in Anwendungen mit strengen Latenzanforderungen.
