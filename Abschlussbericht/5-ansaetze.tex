\section{Ansätze}
% Welche klassischen, Deep Learning -, und eigenen Ansätze wurde verwendet?
% Kurze Beschreibung neuer Techniken und Ideen

In diesem Abschnitt werden die Ansätze zur Stimmungsanalyse von Tweets beschrieben, die wir verwendet haben.
In den Unterabschnitten \ref{subsec:klassische-ansaetze} und \ref{subsec:deep-learning-ansaetze} werden die verwendeten klassischen Ansätze und \gls{dl} basierten Ansätze aufgezählt und beschrieben.

\subsection{Klassische Ansätze}\label{subsec:klassische-ansaetze}

Als klassische Ansätze zur Stimmungsanalyse von Tweets wurden folgende überwachten Lernverfahren ausgewählt, da diese besonders häufig in der Literatur zu finden sind \cite{medhat2014sentiment, wankhade2022survey, zimbra2018state}:

\begin{itemize}
    \item Logistische Regression
    \item \gls{svm}
    \item Naiver Bayes Klassifikator
\end{itemize}

Die Verfahren \gls{knn} und Entscheidungsbäume wurden, mit dem Wissen, dass diese vermutlich nicht für die Stimmungsanalyse geeignet sind, ebenfalls überprüft, aber verworfen, da diese auf dem von uns verwendeten Datensatz schlechtere Ergebnisse erzielten und darüber hinaus längere Laufzeiten benötigten.

\subsection{Deep Learning Ansätze}\label{subsec:deep-learning-ansaetze}

Nach Wankhade et al. \cite{wankhade2022survey} haben in den letzten Jahren \textit{Ţransformer} basierte Modelle \gls{lstm} und \gls{cnn} Modelle in der Stimmungsanalyse abgelöst.
Aus diesem Grund haben wir uns für die Verwendung von \textit{Transformer} basierten \gls{llm} entschieden.

Es wurden \gls{bert}-basierte und \textit{DeepSeek}-basierte Modelle als Vertreter der \textit{Transformer}-Modelle ausgewählt.\\
\gls{bert}-basierte Modelle wurden ausgewählt, da diese in der Literatur als besonders erfolgreich in der Stimmungsanalyse beschrieben werden \cite{devlin2018bert}.
Die \gls{bert}-basierten Modelle und Ansätze sind in Unterabschnitt \ref{subsec:bert} beschrieben.\\
Die \textit{DeepSeek}-R1-basierten Modellen wurden im Rahmen des eigenes Ansatzes aufgrund ihrer Aktualität und der allgemein guten Ergebnisse für unsere Problemstellung evaluiert.
Die Modelle und Ansätze sind in Unterabschnitt \ref{subsec:deepseek} beschrieben.

\textit{Transformer} basieren auf der von Vaswani et al. beschriebenen Architektur \textit{Multi-layer Bidirectional Transformer Encoder} \cite{vaswani2017attention}.

\paragraph{\textit{Transformer}-Architektur}
Die Architektur eines \textit{Multi-layer Bidirectional Transformer Encoder} besteht aus einen \textit{Encoder}, der eine Eingabe von Symbolen $(x_1,...,x_n)$ auf eine kontinuierliche Repräsentation $(z_1,...,z_n)$ abbildet und einem \textit{Decoder}, der aus dieser kontinuierlichen Repräsentation eine Ausgabesequenz aus Symbolen $(y_1,...,y_n)$ generiert.
In jedem Schritt verwendet der \textit{Decoder} seine Ausgabe aus dem vorherigen Schritt.
\textit{Encoder} besteht aus $N$ gleichen Schichten, die jeweils aus einem \textit{Multi-head Self-Attention}-Mechanismus und einem voll-vernetzten Neuronalen Netz bestehen.
Der \textit{Decoder} besteht aus $N$ gleichen Schichten, bestehend aus einer Schicht \textit{Multi-head Self-Attention} und einem voll-vernetzten Neuronalen Netz.
Zusätzlich kommt eine weitere Schicht \textit{Multi-head Self-Attention}, welche die Ausgabe des \textit{Encoders} verarbeitet.

\paragraph{Training von \textit{Transformern}}

Das Training von Transformer-Modellen besteht aus zwei Schritten, dem \textit{Pre-Training} und dem \textit{Finetuning} \cite{Radford2018ImprovingLU}.
Beim \textit{Pre-Training} wird das Modell un-überwacht mit großen Datenmengen trainiert.
Im zweiten Schritt, dem \textit{Finetuning} wird das Modell mit überwachten Lernverfahren oder \textit{Reinforcement Learning} \cite{devlin2018bert, deepseekai2025deepseekr1incentivizingreasoningcapability} auf den jeweiligen Anwendungsfall angepasst.


\subsubsection{\textit{BERT}-basierte Ansätze}\label{subsec:bert}

\gls{bert} ist eine \textit{Transformer}-basierte \gls{llm} Variante, die von Devlin et al. entwickelt wurde \cite{devlin2018bert}.
Devlin et al. haben gezeigt, dass die \gls{bert}-Modelle sehr gute Ergebnisse in der Stimmungsanalyse erzielen und durch \textit{Finetuning} mit vergleichsweise wenig Daten und Trainingszeit auf spezifische Anwendungen angepasst werden kann.
Aufbauend auf \gls{bert} wurde von Liu et al. \cite{liu2019roberta} das Modell \gls{roberta} entwickelt, das durch Optimierungen im \textit{Pre-Training} und \textit{Finetuning} für viele Aufgaben bessere Ergebnisse erzielt.

Von Barbieri et al. \cite{barbieri2020tweeteval} wurde gezeigt, dass für Twitter-spezifische Klassifikationsprobleme \gls{roberta}-basierte Modelle durch \textit{Finetuning} auf Twitter-Daten bessere Ergebnisse erzielen.
Aus diesem Grund haben wir uns für die Verwendung des von Barbieri et al. entwickelten Modells entschieden, welches bereits mittels \textit{Finetuning} an Twitter-Daten angepasst wurde.\\
Für dieses Modell wurden zwei Varianten evaluiert: Eine Verwendung ohne Anpassung auf dem Testdatensatz und eine Anpassung mittels \textit{Finetuning} auf dem Trainingsdatensatz mit anschließender Evaluation auf dem Testdatensatz.

Zum Vergleich haben wir das von \gls{bert} destillierte Modell \textit{DistillBert} \cite{sanh2019distilbert} verwendet.
Bei der Modelldestillation wird ein kleines Modell darauf trainiert, das Verhalten eines großen Modells zu replizieren \cite{sanh2019distilbert}.
Auf diesem Modell wurde ebenfalls ein \textit{Finetuning} auf dem Trainingsdatensatz und eine Evaluation auf dem Testdatensatz durchgeführt.


\subsubsection{\textit{DeepSeek}-basierte Ansätze}\label{subsec:deepseek}

Als eigenen Ansatz haben wir verschiedene vom \textit{DeepSeek}-R1-Modell destillierte Modelle \cite{deepseekai2025deepseekr1incentivizingreasoningcapability} verwendet.
Diese wurden in einem ersten Ansatz versucht, mit \textit{Finetuning} auf unseren Datensatz anzupassen.
Aufgrund der hohen Anforderungen an die Hardware und der langen Trainingszeiten haben wir uns für einen zweiten Ansatz entschieden\footnote{Das \textit{Finetuning} konnten wir lediglich für das kleinste destillierte Modell ausführen.}.\\
In einem zweiten Ansatz haben wir die \textit{DeepSeek}-R1-Modelle mittels \textit{Ollama}\footnote{https://ollama.com/} ausgeführt und mittels Prompt-Anfragen eine Klassifizierung der Tweets durch die Modelle durchgeführt.
Die Prompt-Anfragen wurden dabei mit und ohne Verwendung des Query-Ausdrucks des Testdatensatzes ausgeführt.
