\section{Ausblick}\label{sec:ausblick}

Es gibt mehrere Ansätze, um die Vorgehensweise und die Ergebnisse der Stimmungsanalyse zu optimieren.

\subsection{\textit{Noisy Labels}}
Wie in Abschnitt \ref{subsec:trainingsdatensatz} beschrieben, führt die Art, wie der Trainingsdatensatz \textit{Sentiment140} erstellt wurde, zu \textit{Noisy Labels}.
Dies kann bei der Verwendung einfacherer Modelle zu schlechteren Klassifikationsergebnissen führen sowie die Anzahl der benötigten Trainingsbeispiele oder die Komplexität der Modelle erhöhen \cite{NoisyLabel2014}, wie man auch an den moderaten Ergebnissen der verwendeten klassischen Verfahren ($\approx{85}\%$) und den Resultaten von Go et al. ($79-83\%$) sieht (vgl. Tabelle 6 in \cite{go2009twitter}).

Um mit \textit{Noisy Labels} umzugehen, können robustere Modelle oder Methoden der Datenbereinigung verwendet werden, also Datensätze entfernt oder sogenannte semi-überwachte Lernverfahren angewendet werden. Alternativ können auch \textit{Label Noise} tolerante Lernverfahren angewandt werden (vgl. Abschnitt 3 in \cite{NoisyLabel2014}).

Eine grundsätzlich andere Herangehensweise ist die Verwendung eines Datensatzes mit höherer Datenqualität, um \textit{Noisy Labels} zu vermeiden.

\subsection{\textit{Aspect Based Sentiment Analysis}}

Die erlangten Ergebnisse der unterschiedlichen Modelle können durch die Verwendung zusätzlicher Informationen verbessert werden, wie die Ergebnisse der Experimente mit den \textit{DeepSeek}-basierten Modellen zeigen.

Bei der \textit{Aspect Based Sentiment Analysis} werden explizite oder implizite Begriffe (\textit{aspects}) aus dem Tweet extrahiert und die Stimmung des Tweets bezüglich dieser Begriffe klassifiziert \cite{Hua_2024}.

Ein Beispiel aus \cite{Hua_2024} zeigt, dass der Satz \glqq \textit{The restaurant was expensive, but the menu was great.}\grqq{} bezüglich dem expliziten Begriff \textit{menu} positiv und bezüglich dem impliziten Begriff \textit{price} negativ klassifiziert wird.

\subsection{Große \textit{Deep Learning} Modelle}

Die Feinabstimmung großer \textit{DeepSeek}-Modelle mit dem \textit{Sentiment140}-Datensatz stieß auf Rechenkapazitätsgrenzen.
Während das Modell \textit{DeepSeek-R1-1.5B} noch abgestimmt werden konnte, waren die Speicheranforderungen während des Trainings für die größeren Modelle mit 8, 32 oder 70 Mrd. Parametern zu hoch für die zur Verfügung stehenden Rechenressourcen.

Mehr Rechenkapazitäten oder Methoden wie die \textit{Low Rank Adaptation} (LoRA) \cite{lora2021}, bei der nur eine kleine Anzahl neuer Parameter zur Feinabstimmung genutzt wird, können hier Abhilfe schaffen.
