%%% Choose between 16:9 and 4:3 format by commenting out/uncommenting one of the following lines:
\documentclass[aspectratio=169]{beamer} % 16:9
% \documentclass{beamer} % 4:3

%=========================================================================================================================

\usepackage[ngerman]{babel} % Für neue deutsche Rechtschreibung
\usepackage[utf8]{inputenc}
\usepackage{tikz}               % For creating graphics
\usepackage{url}                % For including urls
\usepackage{tabularx}           % For better tables
\usepackage{colortbl}  % Für \rowcolors und farbige Tabellen
\usepackage{array}     % Für zusätzliche Tabellen-Layouts
\usepackage{xcolor}    % Für Farben

\usetheme{aig}                  % Set beamer theme

%=========================================================================================================================
\title{Stimmungsanalyse mit Twitter}

\author[Team Twitter Sentiment]{Anne Huber, Andreas Franke, Felix Lindner, Burak Özkan, Milomir Soknic}
\institute{Projektpraktikum Web Science,\\Artificial Intelligence Group,\\
University of Hagen, Germany}
\date{17. Dezember 2024}
%=========================================================================================================================
\logo{\includegraphics[width=3cm]{figures/logoaig.png}}
%=========================================================================================================================

\begin{document}

%=========================================================================================================================

\begin{frame}
  \titlepage
\end{frame}
\nologo

\begin{frame}{Motivation}
  \Large
  \begin{itemize}
      \item Stimmungsanalyse mit Twitter
      \item Besonderheiten der Tweets
      \item Relevanz zu Web Science
  \end{itemize}
  \vspace{0.5cm} % Fügt einen vertikalen Abstand ein
  \textbf{Zielsetzung}
  \begin{itemize}
      \item Vergleich unterschiedlicher ML-Ansätze
  \end{itemize}
\end{frame}

%=========================================================================================================================
\section{Daten}


\begin{frame}{Datenauswahl}
  \Large
  \begin{itemize}
      \item Prüfung diverser Datensätze 
      \item Entscheidung für \glqq Sentiment140\grqq
      \item Besonderheiten:
      \begin{itemize}
          \item Emojis als Sentiment-Indikatoren
          \item Ausbalancierte Klassen
          \item Bessere Datenqualität
          \item Artikel: \glqq Twitter Sentiment Classification using Distant Supervision\grqq
      \end{itemize}
  \end{itemize}
\end{frame}

\begin{frame}{Datenaufbereitung}
  \Large
  \begin{itemize}
      \item Bereinigung
      \item Tokenisierung
      \item Transformation
      \item Merkmalsextraktion
  \end{itemize}
\end{frame}

%=========================================================================================================================
\section{Methodik}

\begin{frame}{Klassische Methoden}
  \Large
  \begin{itemize}
      \item Logistische Regression
      \item \textit{Support Vector Machine}
      \item Entscheidungsbäume
      \item Naiver Bayes Klassifikator
      \item K-nächste Nachbarn
  \end{itemize}
\end{frame}

\begin{frame}{Deep Learning}
  \Large
  \begin{itemize}
      \item \glqq RoBERTa\grqq
      \item vortrainiert
      \item BERT-basiert
      \item bildet die Referenz
  \end{itemize}
\end{frame}

%=========================================================================================================================
\section{Ergebnisse}

\begin{frame}{Ergebnisse}
  \Large
  \begin{tabularx}{\textwidth}{|X|c|}
      \hline
      \rowcolor{aigblue!40} % Kopfzeile
      \textcolor{white}{\textbf{Algorithmus}} & \textcolor{white}{\textbf{Genauigkeit}} \\ \hline
      K-nächste Nachbarn & 0.68 \\ \hline
      Entscheidungsbäume & 0.78 \\ \hline
      Logistische Regression & 0.79 \\ \hline
      Naiver Bayes Klassifikator & 0.83 \\ \hline
      \textit{Support Vector Machine} & 0.84 \\ \hline
      \rowcolor{aigyellow!30} % Baseline-Hervorhebung
      \textbf{Twitter-RoBERTa-BaseSentiment } & \textbf{0.87} \\ \hline
  \end{tabularx}
\end{frame}

%=========================================================================================================================
\section{Zusammenfassung}

\begin{frame}{Zusammenfassung}
  \Large
  \begin{itemize}
      \item Datensatzauswahl und Bereinigung
      \item Training klassischer ML-Verfahren
      \item Tests zeigen bessere Ergebnisse mit \textit{Deep Learning} Ansatz
      \item Ausblick:
      \begin{itemize}
          \item Optimierung der Modelle
          \item \textit{Deep Learning} Ansatz fokussieren
      \end{itemize}
  \end{itemize}

  \vspace{1cm}
  \Large
  \pause Vielen Dank für Ihre Aufmerksamkeit!
  \vspace{0.5cm}
  \textit{Offene Fragen?}
\end{frame}


\end{document}